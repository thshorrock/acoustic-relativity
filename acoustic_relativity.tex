\documentclass[10pt, fleqn,final,showtrims,oldfontcommands]{article} %twoside,%\documentclass[10pt, astron]{article} %when have admin rights,astron


\usepackage{emp,ifpdf}
\ifpdf
 \DeclareGraphicsRule{*}{mps}{*}{}
\fi

\usepackage[sort&compress,square, super]{natbib}

\title{An acoustic special theory of relativity}

\author{Tom H. Shorrock}

\graphicspath{{figs/}} 
%\setlength{\headheight}{22.55pt}
\setlength{\parindent}{0mm}
\setlength{\parskip}{2mm} 


\usepackage{graphicx}
\usepackage[mediumspace , textstyle, amssymb ]{SIunits} %change \mu to \micro
\usepackage{subfig}
\usepackage{amsmath, amsthm}


\usepackage{marginnote}
\usepackage{todonotes}


\newcommand{\sub}[1]{\begin{subequations}#1\end{subequations}}
 \newcommand{\subl}[2]{\begin{subequations}\label{eqn:#2}#1\end{subequations}}
 \newcommand{\eqa}[1]{\begin{align}#1\end{align}}
 \newcommand{\eqal}[2]{\begin{align}#1\label{eqn:#2}\end{align}}

\newcommand{\secref}[1]{section~\ref{sec:#1}}
\newcommand{\Secref}[1]{Section~\ref{sec:#1}}
\newcommand{\eqnref}[1]{\ref{eqn:#1}}
\newcommand{\Eqnref}[1]{\ref{eqn:#1}}
\newcommand{\figref}[1]{Figure~\ref{fig:#1}}
\newcommand{\Figref}[1]{Figure~\ref{fig:#1}}

\newcommand{\epsillon}{\epsilon}

\newcommand{\lr}[1]{\left( #1 \right)}
\newcommand{\lrsquare}[1]{\left[ #1 \right]}
\newcommand{\dprime}{{\prime\prime}}
\renewcommand{\d}{\partial}
\newcommand{\del}{\nabla}
\newcommand{\vdel}{ \mbox{\boldmath $\del$}}
\newcommand{\abs}[1]{\left| #1 \right|}
\newcommand{\given}[2]{ \left.{#1}\right|_{#2}  }

\newcommand{\tm}{\tau^-}
\newcommand{\tmp}{\tau^{\prime-}}
\newcommand{\tp}{\tau^+}
\newcommand{\tpp}{\tau^{\prime+}}
\newcommand{\dr}{{\d_r}}
\newcommand{\dt}{{\d_t}}
\newcommand{\dtau}{{\d_\tau}}
\newcommand{\Tr}{\textrm{Tr}}


\newcommand{\scope}[1]{\tilde{#1}}
\newcommand{\half}{\tfrac{1}{2}}

\newcommand{\aether}{\ae ther}
\newcommand{\aetherial}{\ae therial}
\newcommand{\Poincare}{Poincar{\'e}}

\newcommand{\chapref}[1]{chapter~\ref{ch:#1}}
\newcommand{\Chapref}[1]{Chapter~\ref{ch:#1}}


\newcommand{\Tgi}{T_{\textrm{gi}}}
\newcommand{\OmegaEM}{\Omega_{\textrm{EM}}}

 %\newcommand{\tm}{\tau^-}
 %\newcommand{\tp}{\tau^+}

 % \newcommand{\vx}{\vect{x}}
  \newcommand{\vf}{\vect{f}}
  \newcommand{\J}{{\cal J}}
 \newcommand{\vJ}{\vect J}
 % \newcommand{\vj}{\vect j}
  \newcommand{\vE}{\vect E}
  \newcommand{\vB}{\vect B}
 \newcommand{\vEnr}{\vect{E}_{\textrm{nr}}}
 \newcommand{\vBnr}{\vect{B}_{\textrm{nr}}}
% \newcommand{\va}{\vect a}
% \renewcommand{\vA}{\vect A}
% \newcommand{\jp}{\pi}
 \newcommand{\M}{\alpha}
 \newcommand{\vP}{\vect P}
%\newcommand{\vf}{\vect f}
% \newcommand{\vF}{\vect F}
% \renewcommand{\AA}{\mathbbm{A}}

 \newcommand{\Dt}{D_t}
% \newcommand{\fbar}{\underline{f}}
 \newcommand{\adjoint}[1]{\bar{#1}}
% \newcommand{\fadj}{\adjoint{f}}


\newcommand{\scalar}[1]{\left< #1 \right>}
\newcommand{\mono}[1]{\left< #1 \right>_1}
\newcommand{\bi}[1]{\left< #1 \right>_2}
\newcommand{\tri}[1]{\left< #1 \right>_3}
\newcommand{\pseudo}[1]{\left< #1 \right>_4}
\newcommand{\multi}[2]{{\left< #1 \right>_{#2}}}
\newcommand{\bivector}[1]{{\left< #1 \right>_2}}

\newcommand{\eff}{{\textrm{eff}}}

\newcommand{\g}{\gamma_0}
\renewcommand{\H}{{\cal H}}

\newcommand{\vect}[1]{\mathbf{#1}}
\newcommand{\vu}{\textbf{u}}
\newcommand{\vA}{\textbf{A}}
\newcommand{\vv}{\textbf{v}}
\newcommand{\nlist}[1]
	   {  
	     \begin{enumerate}
	       #1
	     \end{enumerate} 
	   }
\newcommand{\nlistcompact}[1]
	   {  
	     \begin{enumerate*}
	       #1
	     \end{enumerate*} 
	   }

\begin{document}

\maketitle
\begin{abstract}
Abstract goes here
\end{abstract}

\section{Introduction}\label{sec:introduction}

The privileged role of light is perhaps the most mysterious aspect of Einstein's special theory of relativity.
What is it about this signal, as opposed to any other means of communication, that makes it  so fundamental to the  concepts of time and space?
The answer, for Einstein, was the constancy of the speed of light,
and using this fact with the relativity postulate he eliminated the  various \aetherial\ explanations of electrodynamics that were present in the nineteenth century.
This triumph, however, only emphasises the uniqueness of light, for it is the  only  signal that has a  medium, or {\em \aether}, with no measurable mechanical properties.
%The central role of light in the measurement process is unsettling. 
How lucky we are to be born with eyes! %be the only signal that has no measurable medium - no \aether.
Satisfactory definitions of time and space seem to come only at the expense of making light  an even greater puzzle,
more and more distant from the world that we can touch and hear.
We are very fortunate that we can see.

The relativity theory of \Poincare, on the other hand,  neither gave a privileged role to light nor eliminated the \aether.
Instead, in addition to the relativity postulate, \Poincare\ postulated a  contraction to an object  when it moves with respect to the \aether.
%For Einstein  moving objects are {\em measured} to be smaller.
%For \Poincare\ moving objects {\em are} smaller.
The most jarring aspects of Einstein's theory are absent from \Poincare's theory but only at the cost of introducing motion dependant deformations.
Without a  physical explanation for these contractions
this cure, for many, is worse than the illness.

In this report we use sound to define time and space 
in the manner  routinely used in medical ultrasound and other sonar-based technologies.
It is demonstrated in section~\ref{sec:measurement} that \Poincare's relativity is recovered if the relativity postulate is also assumed.
Furthermore, in section~\ref{sec:Maxwell} it is demonstrated that the acoustics of an ideal fluid, when measured with ultrasound, 
obey Maxwell's relation.
In short, the generation and propagation of sound,  when time and space are defined with  sound,
obeys the same physical law as the generation and propagation of light,  when time and space are defined with light.

\Poincare's motion dependent contraction makes sense in the context of ultrasound if it is taken very literally.
In {\em Science and Hypothesis}\cite{Poincare1902} \Poincare\ declared that ``Experiment is the sole source of truth''.
%We interpret this as \Poincare's {\em definition} of truth.
It is easily demonstrated (\secref{measurement}) that the dimensions of an object as  measured with  ultrasound 
is dependant upon its motion with respect to the bulk flow of the medium.
\Poincare's contraction follows by  demanding that these experimental results be accepted as true.
%There is no need to recourse to problematic \aetherial-forces that were proposed by Lorentz.
It follows that  every truth  must be stated with a caveat that explains how time and space were  measured: 
the measured dimensions of an object moving with respect to the medium   depend upon whether it is measured with light or sound.
It should be emphasised that  proceeding otherwise does not reduce the anthropocentric nature of measurement, 
it merely prejudices sight over all the other senses.




%Using the acoustic definition of space an object moving with respect to the bulk flow is measured as being contracted.
%Likewise the time as measured with ultrasound is dilated.
%The space-contraction and time-dilation result from the influence of the motion with respect to the bulk flow on the measurement process.
%As experimental result are as true  as any other phenomenon - 
%but there is no need to recourse to \aetherial-forces such as those proposed by Lorentz.

%Finally in section~\ref{sec:bubble} the radial pulsations of a micron sized bubble are studied as a specific example.
%The resonances of such bubbles are important in medical ultrasound because they  radiate sound
%and are used to improve contrast.
%At high pressures the surface of the bubble is predicted to collapse and rebound at faster than the speed of sound.
%The  loss of temporal ordering to such events mean that without a relativistic treatment the ultrasound measurements are impossible  to predict.


\section{The acoustic definition of time and space}\label{sec:measurement}


In medical ultrasound distances are measured using the time it takes a pulse of sound to propagate from a transducer
to a reflecting object and then to return again. 
%This interval is known as the pulse-echo time.
If the sound is emitted from the transducer at a time, $\tm$,
and the sound returns at a time,  $\tp$,
then the task is to find from these two numbers the spatio-temporal location, $x$,
of the point of reflection.

What happens to the sound in between leaving the transducer and returning
cannot be known by acoustic measurement.
In this ignorance ultrasound practitioners assume that the time at which the echo 
occurred is the midpoint of $\tm$ and $\tp$,
\sub{
\label{eqn:radar}
\begin{align}
 \tau(x) &= \frac{\tp + \tm}{2}.\label{eqn:radarTime}
\intertext{Other choices could certainly be made, 
  but would imply a knowledge of the world beyond that learnt from $\tm$ and $\tp$ alone.
  To measure distances from the times $\tm$ and $\tp$ a sound speed, $c$, is required.
  Assuming, again in ignorance, that the sound returns at the same speed at which it left gives
}
 \rho(x) &= \frac{\tp - \tm}{2}c. \label{eqn:radarDistance}
\end{align}
}
These are the definitions of time and space that are used in ultrasound.
They are identical to definitions of time and space used by \Poincare\ and Einstein,
with the exception that the speed $c$ is here the speed of sound rather than the speed of light.
%The assumption that the sound returns at the same speed as it left may be relaxed.
%Then $\tau(x) =\epsilon \tp + (1-\epsilon)\tm$ and 
%$\rho(x) = \epsilon \tp c - (1-\epsilon)\tm c$ for any
%$0<\epsilon<1$.
%This does not change any of the arguments that follow\cite{Debs1996}.

Equation \eqnref{radarDistance} requires an {\em a priori} knowledge of the sound speed
for otherwise distances cannot be determined from temporal measurements.
In diagnostic ultrasound scanners it is usually taken to be \unit{1540}\metre\reciprocal\second.
%This is of course obvious because ultrasound attempts to determine distances from temporal measurements.
%Nevertheless, 
%it does raise the question as to how the sound speed is to be found.

\subsection{Physics as measured with ultrasound}

The measurement rules of equations~\ref{eqn:radarTime} and \ref{eqn:radarDistance} enable two properties of the world as measured by ultrasound to be stated immediately.
The first is that an entity that moves away from the transducer at a speed that is faster than the speed of sound (with respect to the bulk flow of the medium) 
cannot be measured.  
The sound will never catch  up with the entity and so it will never scatter the sound back to the transducer.

The second is that ultrasound is not capable of  measuring variations in the speed of sound.
In order to measure distances (and therefore speeds) the speed of sound must be known {\em a priori}.
Fluctuations in the sound speed  cannot be known without further {\em a priori} knowledge of the medium. 
Fluctuations in the density  can therefore play no role in determining the sound speed. % when ultrasound is used to describe the world.
It follows that sound must propagate according to a linear wave equation. % when the acoustic definitions of time and space are used.
In \secref{Maxwell} it will be demonstrated that this linear relation is identical to Maxwell's relation.

The ultrasound literature does not comply with these deductions.
Currently, when describing the physics of ultrasound, a fluid medium is always described by a Galilean invariant theory such as Euler's equation or Naiver-Stokes' equation.
This enables motions that are faster than the speed of sound to be tracked 
and predicts that  a sound pulse  propagates according to  a non-linear wave equation.
Both of these predictions are impossible when the world is measured with sound.
%Such a description is only meaningful only when distances and times are to be defined with light, 
%which travels at such a tremendous speed that the Galilean approximation is appropriate.%
%
%This is appropriate only if distances and times are to be defined with light, which travels at such a tremendous speed that the Galilean approximation is appropriate.
%If distances and times are to be defined with sound, as is done by the ultrasound scanner,
%then the Galilean description of the world is meaningless, in the sense that it predicts  results that are impossible to verify.
Currently ultrasound physics fails to recognise the distinction between  two equally valid descriptions of the world -
the world that is seen
and the world that is heard.
In doing so ultrasound physics repeats the curious  fallacy that  the world must be seen to be believed.

\subsection{An acoustic Michelson-Morely  experiment.}

It is useful to compare the two viewpoints - the Galilean%
\footnote{Formally the `Galilean' measurements are the distances and times that are  measured with light signals in accordance to Einstein's method.
In ultrasound experiments, however, the Galilean approximation is entirely appropriate.}
 world that is {\em seen} 
and the world that is measured with ultrasound - on a simple pulse echo experiment.
%To do so, an acoustic version of the  Michelson-Morely experiment is considered.
%For this we %
%
% apply the acoustic definitions of time and space, equation~\ref{eqn:radar}, to a simple example.
%This is described in \secref{MMsetup}.

%The analysis of the experiment is done in two parts.
%The first, in \secref{MMGalilean}, is the from the perspective of a `Galilean' observer that {\em looks} at the setup,
%measuring distances with a ruler and times with a single oscilloscope.
%Formally the `Galilean' measurements are the distances and times that are  measured with light signals in accordance to Einstein's method.
%In ultrasound experiments, however, the Galilean approximation is entirely appropriate.
%This is the usual viewpoint from which ultrasound experiment are described.

%The second perspective, in \secref{MMLorentzian}, is taken by considering only  acoustical measurements - the pulse echo times $\tm$ and $\tp$ -
%and using the definitions of equation~\ref{eqn:radar} to determine measured spatial and temporal locations of distant entities.
%This is the viewpoint from which ultrasound experimental data is usually collected.

%\subsubsection{The experiment}




% Since acoustic waves propagate so much more slowly than light,
% an acoustic Michelson-Morely type experiment need not be an interferometry experiment.
% Rather the time it takes for a short pulse to propagate and return can be measured directly.


% The (hypothetical) apparatus  that is to be used in this discussion is illustrated in \figref{MMapparatus}.
% The transducer is capable of generating and receiving an acoustic pulse.
% The central acoustic reflector can be opened or closed 
% The  acoustic source/reciever

% The acoustic source and acoustic receiver are two separate devices that are mounted onto a rail 
% on which both the source and receiver may be translated.
% Both the source and receiver are directional in that they emit and receive in the forward direction only.
% The sound emitted is a short burst.
% Both the source and receiver are facing  an acoustic reflector that is placed in the parallel to the rail on which the source and receiver  translate.
% The shortest distance between the reflector and the rail denoted $l$.
% The whole setup may be rotated, and it is placed in an ideal fluid to propagate the sound.

\subsubsection{A Galilean observer}\label{sec:MMGalilean}

 \begin{figure}[t]
      \centering
     \subfloat[Apparatus when bulk medium is stationary.]{
           \label{fig:setupA}
           \includegraphics{Michelson.0}}
\hfill
     \subfloat[Apparatus when the bulk flow of the medium is $v$]{
           \label{fig:setupB}
           \includegraphics{Michelson.1}}
      % \subfloat[\unit{100}\kilo\pascal]{
      %      \label{fig:R1vel}
      %      \includegraphics{velocity_r2_f2_a0.1.3}}
\label{fig:setups}
      \caption{Apparatus with an without a bulk laminar flow}
 \end{figure}
The first case to be considered is illustrated in \figref{setupA}.
This apparatus is appropriate when the equipment is stationary with respect to the bulk flow of the medium.
It is  analogous to  Michelson and Morely's famous experiment:
a piezoelectric transducer  replaces the light source and a medium that partially reflects sound has replaced the semi-silvered mirror.
The distance between $A$ and $B$ is denoted $l$ and is the same distance as between $A$ and $C$.
In the following the time it takes for the sound to propagate from $A$ to $B$ and back again is compared with the to-fro times between $A$ and $C$.


If the apparatus of \figref{setupA} were not stationary with respect to the bulk flow of the medium then the experiment would fail.
This is because the sound would not travel from  $A$ to $B$ and return.
The motion of the medium would drag the sound pulse with it.
The setup illustrated in \figref{setupB} gives spirit of the Michelson experiment for the case when the apparatus is not stationary with respect to the bulk flow.
In this case there are two separate partially reflecting surfaces.
The time it takes the sound to propagate from $A$ to $B$ to $A^\prime$ is now compared with the time it takes the sound 
to go from $A$ to $C$ to $A^\prime$.

When the to-fro times along the two arms are the same, irrespective of the flow  of the bulk medium,  the result is described as {\em null}.

 \begin{figure}[t]
      \centering
     \subfloat[Observer stationary with respect to medium.]{
           \label{fig:StatGA}
           \includegraphics{Michelson.2}}
\hfill
     \subfloat[Observer moves at speed $v$ with respect to medium]{
           \label{fig:MovingGA}
           \includegraphics{Michelson.3}}
      % \subfloat[\unit{100}\kilo\pascal]{
      %      \label{fig:R1vel}
      %      \includegraphics{velocity_r2_f2_a0.1.3}}
\label{fig:GalileanA}
      \caption{Observed motion of apparatus for the setup of \figref{setupA}}
 \end{figure}

First we consider the case of the apparatus being stationary with respect to the bulk flow (\figref{setupA}).
If the propagation of the sound pulse were observed by a Galilean observer that is also stationary with respect to the flow
then they would observe the sound travelling according to \figref{StatGA}.
The time, $t_{AB}$, it takes for the sound to propagate from the $A$ to $B$ is the same as the time, $t_{BA}$, it takes the sound to propagate from $B$ to $A$.
It is given by $l/c$, where  $c$ is the speed of sound for the medium.
This time interval is the same for the to and fro paths between $A$ and $C$ also,
\begin{align}
  t_{AB}=t_{BA}=t_{AC}=t_{CA}=l/c\label{eqn:setupA:stationary:Tab}.
\end{align}

An observer for whom  both the medium and apparatus flow past at a speed $v$ will measure the same time intervals 
but witness an altogether more complicated experiment.
The acoustic paths that will be observed by this observer are illustrated in \figref{MovingGA}.
When the sound travels between $A$ and $B$ the observer will record that the sound travels at an effective speed of
\begin{align}
\label{eqn:ceffone}
c_\eff(v) = \sqrt{c^2 +v^2}.
\end{align}
This is due to the additional contribution to the speed given by the  laminar flow.
Additionally, the distance between $A$ and $B$ will be measured to be greater by  $\sqrt{l^2+v^2t_{AB}^2}$.
The increased distance and increased speed cancel so that 
\begin{align}
  \label{eqn:setupA:moving:Tab}
  t_{AB} = t_{BA} = \frac{\sqrt{l^2+v^2t_{AB}^2}}{\sqrt{c^2 +v^2}} = \frac{l}{c},
\end{align}
as before.

For the moving observer the bulk flow will also contribute to the effective speed of the pulse from $A$ to $C$  ($c_\eff = c+v$) 
and hinder  the return from $C$ to $A$ ($c_\eff = c-v$).
However, this is again exactly compensated by changes in the total distance that the moving observer measures.
As is illustrated in \figref{MovingGA}, the effective distance from $A$ to $C$ is $l+vt$. % when the pulse passes $A$, the position $C$ is seen to be moving away at a speed of $v$ - the speed of the bulk flow - 
%and so effective distance between $A$ and $C$ is $l+vt$.  
When travelling from $C$ to $A$ the effective distance is $l-vt$. % because the sound pulse is now travelling the opposite direction.
%%
%
%the travelling from  $A$ to $C$ is also viewed by this observer as having further to travel: a distance of $l+vt$.
%The return journey is looks shrunk for this observer, a distance of $l-v$.
Therefore the measured times are
\begin{align}
  \label{eqn:setupA:moving:Tac}
  t_{AC} =  \frac{l+vt}{c+v}= t_{CA} =  \frac{l-vt}{c-v}= \frac{l}{c}.
\end{align}

Next, we must check that these timings still hold when the apparatus is moving with respect to the medium (\figref{setupB}).
%The analysis  proceeds similarly to before.
The equivilance of \figref{setupB} and \figref{MovingGA} demonstrate this.
An observer that is stationary with respect to the apparatus (and moving with a speed $v$ with respect to the medium) will record,
\begin{align}
  \label{eqn:setupB:moving:Tab}
  t_{AB} = t_{BA^\prime} =  \frac{\sqrt{l^2+v^2t_{AB}^2}}{\sqrt{c^2 +v^2}} = \frac{l}{c},
\end{align}
and 
\begin{align}
  \label{eqn:setupB:moving:Tac}
  t_{AC} =  \frac{l+vt_{AC}}{c+v}= t_{CA^\prime} =  \frac{l-vt_{CA^\prime}}{c-v}= \frac{l}{c}.
\end{align}
Equations~\ref{eqn:setupB:moving:Tab} and \ref{eqn:setupB:moving:Tac}  are exactly the same results as equations~\ref{eqn:setupA:moving:Tab} and \ref{eqn:setupB:moving:Tac},
respectively.
If the observer is instead stationary with respect to medium then it is easy to see that equations~\ref{eqn:setupA:stationary:Tab}  are repeated.

In summary, we find that the time it takes the sound to propagate from $A$ to $B$ and back again is
identical to the time it takes the sound to propagate from $A$ to $C$ and back,
irrespective of the speed of the observer with respect to the medium.
The acoustic Michelson-Morely experiment, therefore, should yield a  {\em null} result.
%A {\em null} result  the acoustic Michelson-Morely experiment is correct.
\subsubsection{An acoustic interpretation}\label{sec:MMLorentzian}

When distances are measured acoustically the previous experiment is very difficult to interpret.
This is because the effective sound speed used in equation~\ref{eqn:ceffone}  is contradictory to the acoustic definition of time and space given in equation~\ref{eqn:radar}.
If ultrasound data is all you have then you cannot define a  speed of sound that is a function of the relative velocity of the bulk medium;
you needed the speed of sound {\em before} you could determine the relative speed.


%A physicist using  ultrasound measurements may determine their speed relative to the bulk flow by using \eqnref{radar},
%but only after a constant speed of sound has been assumed.
%It is impossible to define an effective speed of sound that is a function of this relative velocity.

The physicist attempting to infer the passage of the sound through the equipment %(unlike the Galilean observer they cannot directly measure it)
 would  draw a figure such as \figref{MovingGA}.
However, unlike the Galilean observer they cannot directly measure the propagation of the sound.
The inference of the ultrasound physicist is subject to the rules of their measurement system - 
the speed of sound is fixed - and so they would write equation~\ref{eqn:setupA:moving:Tab} as 
\begin{align}
  \label{eqn:setupA:moving:Tab:acoustic}
  t_{AB} = t_{BA} =  \frac{\sqrt{l^2+v^2t_{BA}^2}}{c} \implies t_{AB} = \frac{1}{\sqrt{1-v^2/c^2}} \frac{l}{c}.
\end{align}
%In this case $t_{AB}$ is larger than before but only by a small amount.
For the sound pulse between $A$ and $C$  they would infer
\sub{
\label{eqn:setupA:moving:Tac:acoustic}
\begin{align}
 t_{AC} =  \frac{l+vt_{AC}}{c}\implies t_{AC} = \frac{l}{c-v}
\end{align}
and 
\begin{align}
 t_{CA} =  \frac{l-vt}{c} \implies t_{CA} = \frac{l}{c+v}
\end{align}
}
rather than  equation~\ref{eqn:setupA:moving:Tac}.
The total to-fro time between $A$ and $C$ would then be inferred to be
\begin{align}
t_{AC}+t_{CA} = \frac{1}{{1-v^2/c^2}} \frac{2l}{c}.
\end{align}

Unfortunately, the interpretations of equation~\ref{eqn:setupA:moving:Tab:acoustic} and \ref{eqn:setupA:moving:Tac:acoustic} do not agree with the measured intervals.
The reassignment $c_\eff \rightarrow c$ made by the ultrasound physicist 
has resulted in predicted time intervals for sound to traverse between $A$ and $B$ and between $A$ and $C$ that are too large
by a factor of $\gamma$ and $\gamma^2$ respectively,
%Unfortunately, both the inferred times for $t_{AB} + t_{AB}$ and $t_{AC}+t_{CA}$ given in \eqnref{} and \ref{} are wrong.
%They are, as comparison with \eqnref{} and \ref{} demonstrates, too large by a factor of $\gamma$ and $\gamma^2$ respectively,
where 
\begin{align}
  \gamma = \frac{1}{\sqrt{1-v^2/c^2}}.
\end{align}
Moreover, the to-fro times between $A$ and $B$ are not predicted equal to the to-fro times between $A$ and $C$,
which is demonstrably false.

This predicament faced by the ultrasound physicist  is, of course, a familiar one.
When moving with respect to the bulk flow of the medium the effective speed of sound is different 
to the speed of sound that is assumed by the ultrasound physicist.
They must compensate this error by rescaling their temporal units.
They need, therefore, to re-write equation~\ref{eqn:setupA:moving:Tab:acoustic} as
\begin{align}
 t_{AB}^\ast+t_{BA}^\ast  &= 2\frac{\sqrt{l^2+v^2t_{AB}^2}}{c},\label{eqn:setupA:moving:Tab:acoustic}
\end{align}
where $t^\ast$ are the {\em inferred} temporal intervals of the physicist using acoustic measurements.
By comparing \eqnref{setupA:moving:Tab:acoustic} with \eqnref{setupA:moving:Tab} it is clear that
\begin{align}
 t_{AB}^\ast+t_{BA}^\ast = \gamma^{-1}\lr{t_{AB}+t_{BA}}.
\end{align}
%When the effective speed of sound  is different to  the speed of sound of the medium
%the acoustic observer has no choice but to re-scale their base unit of time.
That is, when an acoustic observer is in motion with respect to the medium 
they must revert to a {\em local-time} that is not the same as the time measured by a Galilean observer.

%The interpretation of the pulse echo between $A$ and $C$ is more straightforward.
For the ultrasound measurements to match equation~\ref{eqn:setupA:moving:Tac} 
a {\em local} distance measure must also be introduced: %If the time for the sound to travel between $A$ and $C$, 
%In this case the observer knows that the total distance that the sound must travel is $l +vt + l - vt = 2l$.
%Therefore 
%\begin{align}
%  t_{AC}+t_{CA} = \frac{2l}{c},
%\end{align}
%in agreement with \eqnref{}.
%Unfortunately the acoustic observer as already had to concede that since they are in motion with respect to the medium,
%the local-time is the unit of time that agrees with the  experiment.
%To compensate for this in \eqnref{}
%the unit of distance must be scaled by the same factor so that
%is to become
\begin{align}
  t_{AC}^\ast+t_{CA}^\ast = \frac{2l^\ast}{c},
\end{align}
where 
\begin{align}
l^\ast = \gamma^{-1} l.
\end{align}
These are the Lorentz-transformations of space and time that result from \Poincare's special relativity.

When an ultrasound transducer moves with respect to the bulk flow the oscilloscope should be re-calibrated to the local time.
The mistakes of the past are repeated when this is not done.
%When ultrasound physicists fail to re-calibrate their oscilloscopes to a local time 
%when their transducer's move with respect to the bulk flow they are destined to repeat the mistakes of the past.


%It may be noted at this point that ultrasound physicists are not in the habit of 
%re-calibrating their oscilloscopes to a local time.
%Firstly we note 
%Finally, it may be noted that when ultrasound measurements the 
%time axis (measured on the oscilloscope) is rarely re-calibrated to the local time
%of a transducer moving with respect to the bulk fluid.
%All that can be said to this is that where this is true,
%the physicist is destined to predict the wrong physics in 
%by repeating of the mistakes of the past.
%However, in practice, 
%the need for such re-calibrations is rare - 
%for in general the transducer is chosen to be at rest with respect 
%to the bulk flow of the medium.

\subsection{Comparison between acoustic relativity with \Poincare's relativity}



% The Galilean observers in \secref{MMGalilean} were able to watch the path of the acoustic pulse as it propagated through the medium.
% To do this they could, for example use Scheirin camera.
% It therefore made sense to discuss the forward path from $A$ to $C$, say, independently from the path from $C$ to $A$
% and to discuss their differing sound speed, for all these properties are measurable.
% However, in an ultrasound experiment such distinctions are not possible.
% Equations~\ref{radar} assume that the forward propagation time is equal to the return time,
% a consequence of  a single sound speed being declared {\em a priori}.

% With this in mind let us reconsider the setup illustrated in \figref{setupA}, 
% which is  appropriate when the apparatus is stationary with respect to the bulk medium.
% If the observer is also stationary with respect to the apparatus then 
% the {\em total} to-fro times for the two arms are % $t_{AB}+t_{BA}$ and $t_{AC}+t_{CA}$.
% % Using the notation of \eqnref{radarTime} we write the total interval
% \begin{align}
%    t_{AB}+t_{BA}=t_{AC}+t_{CA}=2l/c, \label{eqn:setupA:stationary:Tab:acoustic}
% \end{align}
% where \eqnref{eqn:setupA:stationary:Tab} has been used.

% The physicist analysing this acoustic result must then attempt to {\em infer}
% the propagation of the sound paths using \eqnref{setupA:stationary:Tab:acoustic}
% and the assumed  sound speed, $c$.
% In this case the inference is straight-forward.
% The observer knows that they are rest with respect to the medium and knows their equipment.
% (The observer can measure the speed of the medium because they can measure distances and times by equation~\ref{eqn:radar},
% which in turn is possible because they have assumed a sound speed).
% The observer  therefore knows that the distance for the sound to travel is $2l$
% and the experimentally measured time interval is in exact agreement with the assumed speed.



% %The interpretation invited by the right-hand-side of \eqnref{setupA:stationary:Tab:acoustic} -

% However, when the observer is moving with respect to the bulk flow the inference is less easy.
% By using equations~\ref{} and \ref{}  the timings acoustically measured timings are again
% \begin{align}
%   t_{AB}+t_{BA}  &= t_{AC}+t_{CA} = 2l/c.
% \end{align}
% However this time the observer, who knows their equipment and who knows they are moving with a speed of $v$ with respect to the medium 
% (again,  because they have assume a fixed speed of sound and used \eqnref{radar},
% knows that the total distance that the sound must  propagate when moving from $A$ to $B$ and back is $2\sqrt{l^2+v^2t_{AB}^2}$.
% However, the observer has already been forced to assume that the speed of sound is and remains $c$ throughout.
% They therefore write
% \begin{align}
%  t_{AB}^\ast+t_{BA}^\ast  &= 2\frac{\sqrt{l^2+v^2t_{AB}^2}}{c},\label{eqn:setupA:moving:Tab:acoustic}
% \end{align}
% where $t^\ast$ are the {\em inferred} temporal intervals of the physicist using acoustic measurements.
% By comparing \eqnref{setupA:moving:Tab:acoustic} with \eqnref{setupA:moving:Tab} it is clear that
% \begin{align}
%  t_{AB}^\ast+t_{BA}^\ast = \sqrt{1-v^2/c^2}\lr{t_{AB}+t_{BA}}.
% \end{align}
% When the effective speed of sound  is different to  the speed of sound of the medium
% the acoustic observer has no choice but to re-scale their base unit of time.
% That is, when an acoustic observer is in motion with respect to the medium 
% they must revert to a {\em local-time} that is not the same as the time measured by a Galilean observer.

% The interpretation of the pulse echo between $A$ and $C$ is more straightforward.
% In this case the observer knows that the total distance that the sound must travel is $l +vt + l - vt = 2l$.
% Therefore 
% \begin{align}
%   \label{eqn:}
%   t_{AC}+t_{CA} = \frac{2l}{c},
% \end{align}
% in agreement with \eqnref{}.
% Unfortunately the acoustic observer as already had to concede that since they are in motion with respect to the medium,
% the local-time is the unit of time that agrees with the  experiment.
% To compensate for this in \eqnref{}
% the unit of distance must be scaled by the same factor so that
% \begin{align}
%   t_{AC}^\ast+t_{CA}^\ast = \frac{2l^\ast}{c},
% \end{align}
% where $l^\ast = \sqrt{1-v^2/c^2} l$.
% These are the Lorentz-transformations of space and time that result from \Poincare's special relativity.

% \subsection{Discussion}
% The derivation of 



% the acthe acoustic unit of time is 


% medium faced with a faster 

% Using \eqnref{setupA:stationary:Tab:acoustic} with \ref{eqn:radarDistance} we find
% \begin{align}
%   \rho_{AB} \equiv \frac{\lr{t_{AB}+t_{BA}}c}{2}  = l.
% \end{align}
% The acoustic measurement  therefore agrees with tha




% \begin{align}
%    t_{ABA}   \sqrt{1-v^2/c^2} &=2 \frac{\sqrt{l^2+v^2t_{AB}^2}}{c} \\
%   %t_{BA} \sqrt{1-v^2/c^2} &= \frac{\sqrt{l^2+v^2t_{BA}^2}}{c} \\
%   t_{ACA} &= 2\frac{l}{c}
% \end{align}

% This means that when the apparatus of \figref{setupA} are being used only the comparison between 

% In \secref{MMGalilean} it was possible for a Galilean observer to watch the path of the acoustic pulse. observer was able to 

% When discussing what is actually measured acoustically a little care i
% An acoustic observer will interpret the pulse-echo times from carry out the same experiments has 

% Next we reinterpret  the above calculations from the perspective of acoustic observers.
% Remembering that to \eqnref{radar} an acoustic observer must set an {\em a priori} speed of sound.
% They can have no recourse to an effective speed $c_\eff \ne c$.
% Secondly the acoustic observer uses only the total outward and return journey - the difference between $\tp$ and $\tm$.
% % To do so we recognise that an acoustic observer cannot distinguish between the outward and return journey.
% %Therefore, using \eqnref{radarTime}
% %\begin{align}
% %  t_{AB} = t_{BA} = \half t_{ABA},
% %\end{align}
% %always, where $t_{ABA} = t_{AB}+t_{BA}$.

% First \figref{}.
% When the observer is stationary with respect to the bulk flow and the setup then nothing changes in the interpretation:
% \begin{align}
% t_{AB}=t_{BA}=t_{AC}=t_{CA}=l/c.
% \end{align}
% %In this case the outward journey times are predicted to be the same as the return journey times

% When the observer is motiving with respect to the bulk flow there is a problem in interpretation, however.
% For in this case $c_\eff\ne c$ for any of the paths. %This is because an acoustic observer cannot have a locally varying speed of sound, 
% %and the $c_\eff>c$ interpretation to the results cannot be used.
% To resolve this we rearrange ... to find 
% \begin{align}
%   t_{ABA}    \sqrt{1-v^2/c^2} &=2 \frac{\sqrt{l^2+v^2t_{AB}^2}}{c} \\
%   %t_{BA} \sqrt{1-v^2/c^2} &= \frac{\sqrt{l^2+v^2t_{BA}^2}}{c} \\
%   t_{ACA} &= 2\frac{l}{c}
% \end{align}
% These provide measurable  interpretations for the ultrasound experiment, for they are both framed for a constant speed of sound.
% However, the cost of this is to rescale the temporal axis so that 
% \begin{align}
% t^\ast = t_{ABA}    \sqrt{1-v^2/c^2}
% \end{align}
% which in turn demands that 
% \begin{align}
% l^\ast_{ACA} = l_{ACA}    \sqrt{1-v^2/c^2}
% \end{align}
% to compensate.
% But these are the Lorentz transformations, using the \Poincare\ definition of simultaneity.

% %\begin{align}%
% %
% %\end{align}
% %To provide an acoustic interpretation 


% It is necessary to reinterpret 
% Interpreting  $c_\eff = c$ is problematic, however.
% For if we were to do this alone then 
% \begin{align}
%   t_{AB} = t_{BA} =  \half t_{ABA} = \frac{\sqrt{l^2+v^2t_{BA}^2}}{c} \implies t = \frac{1}{\sqrt{1-v^2/c^2}} \frac{l}{c}.
% \end{align}
% %In this case $t_{AB}$ is larger than before but only by a small amount.
% Secondly
% \begin{align}
%  t_{AC} =  \frac{l+vt_{AC}}{c}\implies t_{AC} = \frac{l}{c-v}
% \end{align}
% and 
% \begin{align}
%  t_{CA} =  \frac{l-vt}{c} \implies t_{CA} = \frac{l}{c+v}
% \end{align}
% and so 
% \begin{align}
% t_{AC}+t_{CA} = \frac{1}{{1-v^2/c^2}} \frac{2l}{c}.
% \end{align}
% Using the $\tp-\tm= t_{AB}+t_{BA}$ and $\tp-\tm= t_{AC}+t_{CA}$ in \eqnref{radar} not only predicts the wrong distances for $A$ to $B$ and for $A$ to $C$,
% but also predicts that the two distances are different lengths.
% An acoustic observer  would predict the wrong result to the  Michelson-Morely experiment.

% It is not too hard to see what has gone wrong.
% When the sound travelling from $A$ to $B$ is viewed by an observer moving with respect to the medium 
% the sound {\em really is} travelling at $c_\eff$.  
% To give $c_\eff$ a different value, so that it equates to $c$, 
% a change in base unit is required, not just a blunt reassignment.
% The 






% The situation is not changed if viewed by an observer moving with the bulk flow (but with the setup moving).
% Again a null result is correct for the acoustic Michelson-Morely experiment.

% Rather after leaving $A$ the sound will be carried by the medium towards  because it would be

% when the setup is stationary with respect to the bulk flow of the fluid.

% In this case the acoustic Michelson-Morely experiment can be carried out entirely analogously to the original with light.
% A piezoelectric transducer replaces the light source and acts as the receiver too.
% The partially silvered mirror can be replaced with an angled material of partial acoustic impedance.
% The setup is given in \figref{MMapparatus}.

% It is assumed that the distances from the transducers to the partial reflector,
% and from the partial reflector to the complete reflectors are all $l$.
% When viewed from an observer that moves with the setup (and with the bulk flow of the fluid) 
% the time, $t_{AB}$, it takes for the sound to propagate from the $A$ to $B$ is the same as the time, $T_{BA}$ it takes to propagate from $B$ to $A$.
% Also this time is the same as the to and fro paths between $A$ and $C$.
% For all cases we have
% \begin{align}
%   t_{AB}=t_{BA}=t_{AC}=t_{CA}=l/c,
% \end{align}
% where $c$ is the speed of sound of the medium.

% The same timings are measured when by someone who travels at a speed $-v$ with respect to the bulk flow of the fluid.
% However, when the setup passes this observer with the laminar velocity $v$ their interpretation is slightly different,
% for the sound pulse has the additional contribution of the bulk flow of the medium.
% The sound propagating from $A$ to $B$  has the effective speed, $c_\eff$ of $\sqrt{c^2 +v^2}$.
% However, this sound also has further to travel, the distance from $A$ to $B$ now being $\sqrt{l^2+v^2t^2}$.
% Like wise for the return journey.
% We now have
% \begin{align}
%   t_{AB} = t_{BA} = \frac{\sqrt{l^2+v^2t^2}}{\sqrt{c^2 +v^2}} = \frac{l}{c}.
% \end{align}
% The pulse from $A$ to $C$ is viewed to travel at an effective speed of $c+v$, whereas the return from $C$ to $A$ is viewed to travel at $c-v$.
% However, the pulse from $A$ to $C$ is also viewed by this observer as having further to travel: a distance of $l+vt$.
% The return journey is looks shrunk for this observer, a distance of $l-v$.
% Therefore
% \begin{align}
%   t_{AC} =  \frac{l+vt}{c+v}= t_{CA} =  \frac{l-vt}{c-v}= \frac{l}{c}.
% \end{align}
% This experiment, irrespective of the motion of the observer, produces the same timings.
% A {\em null} result to this half of the acoustic Michelson-Morely experiment is correct.

% The second case to consider is when the setup is moving with respect to the bulk flow of the fluid.
% In this case it is important to recognise that the apparatus of \figref{} will fail.
% The sound will never propagate from $A$ to $B$ as the bulk flow will carry the sound with it.
% To overcome this problem two partial reflectors are required, as drawn in \figref{}.
% The analysis then proceeds similarly to before with,
% \begin{align}
%   t_{AB} = t_{BA^\prime} =  \frac{\sqrt{l^2+v^2t^2}}{\sqrt{c^2 +v^2}} = \frac{l}{c}.
% \end{align}
% and 
% \begin{align}
%   t_{AC} =  \frac{l+vt}{c+v}= t_{CA^\prime} =  \frac{l-vt}{c-v}= \frac{l}{c}.
% \end{align}
% The situation is not changed if viewed by an observer moving with the bulk flow (but with the setup moving).
% Again a null result is correct for the acoustic Michelson-Morely experiment.



% The experiment they witness is drawn in \figref{}.
% For this observer the transducer still emits a pulse that travels in the md
% The velocity of the sound m

% Next we consider the view of another observer, 
% the previous experiment 
% Next we confirm that the same conclusion is drawn by a Galilean observer who 

% Since there is no bulk flow, the emitted sound will return to the receiver only if they are located at the same position along the rail.
% (This could be achieved by having the transmitter being manufactured in two halves through which  the receiver may pass through).
% The time, $t$, it takes the sound to propagate from the transmitter to the reflector is in this case $t = l/c$, where $c$ is the speed of sound of the medium.
% The return time is the same. %total time it takes for the sound to return is $2t=2l/c$.
% This result is the same no matter what angle the setup as a whole is orientated.

% Let us now consider the same experiment but this case in the presence of a laminar flow of speed $v$ in the $x$ direction, say.
% The case the orientation of the setup does matter.

% First we consider the case where the reflector is orthogonal  to the flow, as drawn in \figref{MMperp}.
% When the sound propagates from the source to the reflector it travels with an effective speed of $c_\eff = c+ v$,
% On the return journey the sound travels at an effective speed of $c_\eff = c-v$.
% The time interval for the forward and back journey is therefore $s /$
 
% In addition the sound must travel a distance of $l + vt$ since the 
% In this case the propagation of the sound from the source to the reflector is not the same as from the reflector to th

% If the setup is now rotated so that the reflector is parallel to the flow
% then it is soon apparent that repeating the experiment exactly as before is impossible.
% The flow of the medium carries the sound with it and the emitted sound will  never be received.
% In order to do pulse echo, the transmitter and receiver must be separated by a distance, $2vt$,
% where $t$ is the time interval between the transmitter emitting an acoustic pulse and it arriving at the reflector, as measured by the Galilean observer.
% This modification to the setup is drawn \figref{MMparallel}. 

% In this case, the sound pulse is still emitted towards the reflector at a speed $c$.
% It also has, however, a horizontal component with speed $v$ which is the speed of the bulk flow.
% The effective speed of sound, $c_\eff$, observed in this case is therefore 
% \begin{align}
%   c_\eff = \sqrt{v^2+c^2}.
% \end{align}
% Since the transmitter and receiver are now separated the total distance that the sound must travel is also  increased.
% The distance from the transmitter to the reflector  is
% \begin{align}
%   s =  \sqrt{l^2 + v^2 t^2}
% \end{align}
% The  time of flight for this distance is therefore
% \begin{align}
%   t c_\eff = s/c_\eff = l/c
% \end{align}
% as before.




% The effect of this will be to make the distance $A$ to $B$ seem larger than it is  by a factor of $(1-v^2/c^2)^{-1}$.
% Unfortunately 

% However, the sound does of course  still travel at $c_\eff>c$ and this 

% ot interpret the change in effective speed of sound.

  
% %t_{AB}=t_{BA}=t_{AC}=t_{CA}=l/c,
% It is clear immediately that  important to next understand how the 


% To do so we imagine a transducer and acoustic reflector mounted onto board a distance $l$ apart.
% This distance, $l$,  is informally  the `Galilean' distance measured by {\em looking} at the setup and using a ruler.
% Formally the `Galilean' measurements are the distances and times that are  measured with light signals in accordance to Einstein's method.

% The board  has a pivot below the transducer so that the transducer and reflector may  be rotated together to any angle.
% The board, transducer, and reflector are then placed into an ideal fluid medium.
% The setup is illustrated in \figref{MMsetup}.


% The purpose of this section is to compare the acoustic measurements for the distance and time of the reflection, $\rho$ and $\tau$ respectively,
% with the Galilean position and time of reflection, $r$ and $t$.


% \subsubsection{When the setup is stationary with respect to the bulk flow}
% If the transducer and reflector  is stationary with respect to the bulk flow of the fluid,
% then the Galilean time it takes for the sound to return after a short pulse (the pulse-echo time)
% is 
% \begin{align}
%   2t = 2l/c,
% \end{align}
% no matter what angle the transducer points in the fluid (\figref{MMstationary}).

% If the acoustic clock is set to zero when the pulse was emitted, so that $\tm = 0$, 
% then it follows from \eqnref{radarTime} that the echo occured  at an acoustically measured  time of $l/c$.
% From \eqnref{radarDistance} it follows that the board is measured by ultrasound to be a distance $l$.
% Both the acoustic time and distance measurements agree with the Galilean measurements.

% \subsubsection{When the setup is moving with respect to the bulk flow}

% When  the bulk fluid flows past the transducer and reflector the acoustically measured time and 
% space are not the same as the Galilean measurements.

% This is because the sound get carried along by the bulk flow.

% First consider when the transducer and reflector are parallel to the flow.
% Since the sound gets carried with the bulk flow the Galilean distance that the sound travels is greater than before.
% The path from a Galilean observer (using light signals) is drawn in \figref{MMperpG}.
% The  Galilean distance that the pulse must travel is
% \begin{align}
%  r =  \sqrt{l^2 + v^2t^2},
% \end{align}
% and the pulse echo time case is therefore
% \begin{align}
%  2t = 2r/c =   2\sqrt{l^2 + v^2t^2}/c.
% \end{align}
% It follows that 
% \begin{align}
%    \tp-\tm = \frac{1}{\sqrt{1-v^2/c^2}} \frac{2l}{c}.
% \end{align}

% %The acoustically measured time is therefore
% %\begin{align}
% %  \tau =  \frac{1}{\sqrt{1-v^2/c^2}}  \frac{l}{c}
% %\end{align}
% The acoustically measured distance is
% \begin{align}
%   \rho =  \frac{1}{\sqrt{1-v^2/c^2}} l.
% \end{align}
% However, in contradistinction to the Galilean observer, the ultrasound 


% The rotational invariance is not the case when the medium flows past the setup at a speed $v$.
% In \figref{MMb} for when the transducer and reflector are parallel to the flow,
% and drawn in \figref{MMc} for when the transducer and reflector are perpendicular to the flow.


% In the parallel case:

% The Galilean observer:

% The sound moves with a speed of $c+v$ on the way to the reflector and moves with a speed $c-v$ on the way back from the reflector.
% The total time to travel to and fro from the reflector is therefore
% \begin{align}
%   \tp-\tm = \frac{l}{c+v} + \frac{l}{c-v} = \frac{1}{1+v^2/c^2} \frac{2l}{c}.
% \end{align}
% The acoustically measured distance is therefore,
% \begin{align}
%   \rho = \frac{1}{1+v^2/c^2} l.
% \end{align}


% But the time interval and the space interval must be identical
% so
% \begin{align}
%   dfd
% \end{align}


% The board pivots about transducer is the classic experiment and so we imagine how to ite






% To determine the speed of sound of a given medium an independent notion of length is required.
% This practically would be found with a set of callipers.
% The average speed of sound is then found from equation \eqnref{radarDistance},
% using the times measured and the predetermined ({\em a priori})  distance.
% The point is that %the distance measured by the callipers now represents the {\em a priori} knowledge.
% in order to use equation \eqnref{radarDistance} something must be taken for granted; 
% it is impossible to measure simultaneously both speeds and distances from temporal measurements.

%All this is familiar. %Before ultrasound measurements can be made a calibration measurement to determine the speed of sound is required.
%In order to use ultrasound to measure distances within people (without reaching for the callipers, and therefore for the scalpel)
%it is necessary to first find the appropriate sound speed in the laboratory.
%Bovine liver, typically, is used for this purpose, and is  cut into carefully measured thicknesses,
%with care being taken to known a great deal about the sample before \eqnref{radarDistance} is applied.
%[can you suggest a few  citations for here Jeff...]

%In retrospect the recover of \Poincare's special relativity is not too surprising.
%We have insisted upon the constancy of the propagating signal  and demanded invariance with translations. ...

\todo[inline]{Need to compare this special relativity with \Poincare's SR with Einsteins SR.  I suggest asking Pierseaux to join this article for this - as he is the expert}.

When using equation \eqnref{radarDistance} to measure distances the sound speed must be constant.
This is because there is no way to measure its variations.
This constancy of the speed of sound is identical to Einstein's  second postulate for special relativity\cite{Einstein1905},
except that the sound speed takes the role of the speed of light.
Likewise, Einstein used equations \eqnref{radar} for the definition of time and space, 
except that  the  constant $c$ was understood to be the speed of light rather than the speed of sound.
In the relativity literature equations \eqnref{radar} go by the names of radar-time 
and radar-distance and their application to the example of the ``twin paradox'' is discussed by Dolby and Gull  \cite{Dolby2001}.

In this thesis we  assume that the dynamics of entities is invariant to inertial motions even when measured acoustically (Einstein's first postulate).
There has never been a counter example to this symmetry and we do not expect ultrasound to provide one.
It then follows that measurements in ultrasound are subject to the considerations of special relativity,
where the speed of sound rather than the speed of light takes the role of the limiting velocity.

The surface of a  bubble that has been induced to pulsate by an ultrasound wave may collapse at a significant fraction of the speed of sound\cite{Neppiras1980}.
The pulsations of a bubble as measured by ultrasound are therefore expected to disagree with the same pulsations measured by an optical techniques.
To investigate this, we now derive and analyse a version of the Keller-Miksis model\cite{Keller1980} 
- a commonly used model for a pulsating bubble - that is consistent with how ultrasound measures spatio-temporal locations.
%This will be different than the current Keller-Miksis model that describes the pulsations of a bubble when observed optically under a microscope.


\section{Acoustics when measured with ultrasound}\label{sec:Maxwell}


Ultrasound is not capable of  measuring variations in the speed of sound.
Second order fluctuations in the density  can therefore play no role when ultrasound is used to describe the world.
It is instructive to see this argument borne out in the equations that describe an acoustically measured fluid.
The equations of motion must  be Lorentz invariant and this condition  is automatically fulfilled  when  the equations 
are obtained from the divergence of the energy-momentum tensor of the ideal fluid.
In the derivation we also need to use the hitherto unused  condition
that the sound speed takes the role  of the speed of light.
This condition is imposed by equating the two speeds.
This further requires that the energy density for the fluid, as measured acoustically, be a function of the pressure only (barotripic),
for the sound speed cannot equal the speed of light otherwise\cite{Taub1978}.

The energy-momentum tensor of an ideal fluid is\cite{Doran2003}
\eqal{
  T(a) = w a \cdot u u - a p = (\epsilon + p) a \cdot u u - a p,
}{EMtensor}
where, $\epsilon \equiv \epsilon(p)$ is the barotropic total energy density,
$p$ is the pressure
and 
$u=u(\tau)$ is the velocity vector of the spacetime path, with the parameterisation chosen such that $u^2 = 1$. % and $P$ is the pressure.
Natural units have been chosen in this section, and so that the speed of light is unity.

The speed of sound, $c$,  given at constant entropy density, $\sigma$, is\cite{LandauBook,Taub1978} 
\begin{align}
  c^2 = \given{\frac{\d p}{\d \epsillon}}{\sigma}. \label{eqn:soundspeed}
\end{align}
This is the same as the non-relativistic expression except that the energy density has replaced the mass density.
Setting the speed of sound to equal the speed of light (unity) and the integrating 
 gives,
 \begin{align}
   \epsilon = p^\prime - p_0 + \epsilon_0
 \end{align} 
where $p^\prime$ is a pressure that fluctuates with position, so that $dp = dp^\prime$,
and $p_0$  and $\epsilon_0$  are the ambient  pressure and mean energy density, respectively.
%Rather can carry the constants $ p_0 $ and $\epsilon_0$ through the rest of the derivation we write
The thermodynamic pressure is therefore
\begin{align}
  p \equiv p^\prime - p_0 + \epsilon_0. \label{eqn:pshort}
\end{align}
and the fluid obeys the equation of state, 
\eqal{
  \epsilon(p) = p.
}{eos}
At infinity $p^\prime = p_0$ 
and so $\epsilon_\infty = p_\infty = \epsilon_0 \ne p_0$.

%therefore enforces that the sound speed equals that of light (unity).
%The notation of equation \eqn{eos} is quite compact.
%For \eqnref{eos} to be consistent with the integral of the sound speed (equation \eqnref{soundspeed}),
%the pressure $p$ must be interpreted as follows,
%\begin{align}
%  p = p^\prime - p_0 + \epsilon_0. \label{eqn:pshort}
%\end{align}
%Here $p^\prime$ is the pressure perturbation measured with a hydrophone,
%$p_0$  and $\epsilon_0$  are the mean  pressure and energy density respectively.



Applying \eqnref{eos} simplifies the energy momentum tensor considerably,
\eqal{
  T(a) =  p\lr{2 a \cdot u u  - a} \equiv  \frac{1}{4}  A a A,
}{EMFluid}
where vector potential, $A$, introduced on the right hand side satisfies
\eqal{
  A = 2p^{1/2}u =2 \epsilon^{1/2} u.
}{defnA}

%The vector $A$ is the same as was introduced in \eqnref{RelVPot},
i.e.
\begin{align}
  \del \psi = - \frac{wu}{n} = - 2\sqrt p u.
\end{align}
%where $n$ is the proper particle number density of the fluid and equation \eqnref{eos} has been used for the second equality.
To demonstrate this second equality we  use an argument of Taub\cite{Taub1978}.
Using the isentropic thermodynamic relation $m de = - p d\lr{\frac{1}{n}}$ with the relation for the internal energy, $\epsilon= nm( 1 + e(p))$,
it follows that 
\begin{align}
 n d\epsilon = \epsilon dn - n^2 p d \lr{\frac{1}{n}} = \lr{\epsilon + p} dn.
\end{align}
Applying equation \eqnref{eos} and integrating we obtain
\begin{align}
  n = \sqrt p, \label{eqn:nrootp}
\end{align}
from which we find
\begin{align}
A = 2\sqrt p  u = \frac{2\epsilon}{n} u = \frac{wu}{n} =  - \del \psi,
\end{align}
as claimed.


The divergence of the energy momentum tensor (equation \eqnref{EMFluid}) vanishes in the absence of external fields.
Therefore, by projecting the divergence of \eqnref{EMFluid} along the timelike component we find
\eqa{
  u\cdot\scope T(\scope\del)= \half  u\cdot A \del \cdot A = 0.
}
The check denotes the scope of the derivative.
Since, from \eqnref{defnA}, $A$ is parallel to $u$  it follows that 
\eqal{
  \del \cdot A  =0
}{eomTime}
and so the vector potential $A$ is conserved.
The spacelike projection, $\scope T(\scope \del) - u u\cdot \scope T(\scope\del)$, gives in turn,
\eqal{
  u \cdot \lr{\del \wedge A} = 0.
}{eomSpace}
The relativistic vorticity bivector is 
\eqal{F = \del \wedge A,}{DefnVorticity}
and so \eqnref{eomSpace} implies that the vorticity is orthogonal to the velocity.

By taking the divergence of \eqnref{eomTime} and using the vector identity
$\del \del \cdot A = \del^2 A - \del\cdot\lr{ \del \wedge A} $
 we find,
\eqal{
  \del^2 A = \del \cdot F = \del F.
}{wave}
The second equality follows because $\del F = \del \cdot F + \del \wedge F$ 
and because the  operator identity $\del\wedge \del = 0$ causes $\del \wedge F$ to vanish.
Equation \eqnref{wave} is a wave equation and so interpreting the right-hand-side of \eqnref{wave} as an acoustic source current, $J$, we obtain
\eqal{
  \del  F = J.
}{Maxwell}
Interestingly equation \eqnref{Maxwell} is Maxwell's equation and equation \eqnref{eomTime} has specified that we are working in the Lorenz gauge.
%We do not explore this observation further here.


Acoustics was then formulated for  a ideal and isentropic fluid.
Consistency with the measurement process required the fluid to be relativistic, 
in the sense that is invariant to Lorentz transformations.
To enforce the sound speed to take the role of the speed of light, 
these two constants were equated.
By enforcing the conservation of the energy momentum tensor
(or equivalently, by Noether's theorem, translational invariance)
we found the temporal and spatial equations of motion,
\eqa{
  \del \cdot A  =0 \tag{\ref{eqn:eomTime}}
}
and
\begin{align}
v \cdot \lr{\del \wedge A} = 0, \tag{\ref{eqn:eomSpace}}
\end{align}
respectively,
where c.g.s. units have been  used.
%From this latter condition a to a conserved energy momentum tensor yields
%
%This latter condition yielded the equation of state $p = \epsilon$,
%where $p$ is the thermodynamic pressure and $\epsilon$ is the total energy density.
%Our starting point was the  conservation of the energy momentum tensor,
%which by Noether's theorem is equivalent to demanding translational invariance.
%
From \eqnref{eomTime} we found that it follows that sound waves and acoustic sources are described by Maxwell's relation,
\eqa{
\del F = J, \tag{\ref{eqn:Maxwell}}
}
where 
\eqa{
  F = \del \wedge A\tag{\ref{eqn:DefnVorticity}}
} is the spacetime vortiticy,
and $J$ is the acoustic source in the wave equation $\del^2 A = J$.
The potential $A \equiv 2\sqrt p v$, where $v$ is the velocity of a fluid particle.

The consequences of this correspondence were not explored, however,
and we tie this loose end here.
In \secref{int:EM} and find that 
the role of magnetism is played by the spatial vorticity of the fluid,
and that the electric field corresponds to the Coriolis acceleration.
The space-time vorticity tensor assumes the role of the electromagnetic field tensor.
This completes similar but partial  attempts by others to construct an analogy between acoustics and electromagnetism.
In particular, both Marmanis\cite{Marmanis2000} and Srihar\cite{Sridhar1998}
have suggested and analogy between vorticity and magnetism, 
and the Coriolis acceleration (or Lamb vector) with the electric field.
However, both authors constructed their analogy in the incompressible Galilean frame, 
and therefore could not write down a wave equation (Maxwell's relation).

% For dimensional consistency we find that the acoustic charge must be {\em dimensionless}.
% The analogue to the electric permittivity has dimension $\joule^{-1}\metre^{-1}$.
% We do not have a satisfactory interpretation for the permitivity, however.

% The formal exactness of the analogy presented here implies
% that the Lagrangian for acoustics (as measured with ultrasound)
% is the same form as the electromagnetic Lagrangian.
% In \secref{int:spin} this fact is used with the further assumption of rotational invariance,
% which  by Noether's theorem demands that angular momentum is conserved,
% to show that an acoustically measured sound wave -just as light- has an intrinsic spin.

The equivalence between electromagnetism and acoustics (as measured with ultrasound)
poses several issues of interpretation:
\nlist{
  \item On the one hand sound is a longitudinal wave, the molecules vibrate parallel to the  perturbation. 
    On the other hand, sound is transverse in the Coriolis acceleration and vorticity.
    In what sense are these two views consistent?
  \item Non-linear propagation is known to be important in ultrasound,
    however, in our formulation it is impossible.
    How can this be.
%  \item Rotational invariance implies that sound has an intrinsic angular momentum.
%    Since there are two possible polarisation of the  Coriolis acceleration and vorticity
%    the sound must be spin 1.
%    What does this mean in terms of the dynamics of the fluid particles?
}
Solutions to both these problems are suggested in \secref{Interpretations}.


%interpretations to these are provided in \secref{FutureWork}.
%There is a speculative nature to these interpretations, however,
%which is why they are considered as future work.
%The difficulties in interpretation come not from dought in the validity

%In \secref{int:spin} we impose rotational invariance,
% the consequences of the correspondence between electromagnetism and acoustics.
%For instance it follows immediately  that sound, when measured acoustically,
%can be considered as transverse wave with respect to  the vorticity and Coriolis acceleration.
%If we further assume of rotational invariance, 
%which by Noether's theorem demands that angular momentum is conserved,
%then we find that sound - just as light - has an intrinsic spin.


% An interpretation of the acoustic spin is provided from the helicity.
% The acoustic analogue to the electromagnetic helicity is identical to the hydrodynamic  helicity introduced by Moffatt.
% It is s a topological invariant that describes the linking between vortex rings,
% and is, therefore, quantised.
% The spin of the sound wave describes the propagation of this topological feature through the medium.

% With these results in hand we consider acoustic-vortex interactions in \secref{int:vortex_interactions},
% important when sound interacts with a region of turbulence.
% When the spin plays no role (unlinked vortices in the turbulence) 
% we find that the interactions are described
% by Lorentz force law.
% This linear relation is a very simple and elegant solution to a difficult problem.
% It also highlights the general rule that the  acoustic measurement greatly simplifies the measured physics.
% This is because acoustic measurement intrinsically limits what can be known about the world.

% The exact analogy suggests that there might be an acoustic analogue to the electron: 
% a particle with helicity that interacts with the spin of the acoustic field.
% We do not succeed in finding the analogue, 
% but a start is suggested in \secref{int:electron}.






 

%Sir James Lighthill, in his formulation of acoustics \cite{Lighthill1952},
%gave a general framework for calculating the sound generated by turbulence with a minimum of approximation.


%The influence of boundaries was added to the formulation by Curle \cite{Curle1955} and Ffowcs Williams and Hawkings \cite{FfowcsWilliams1969}.
%More recently,  formulations in terms of the vorticity and enthalpy have also proven useful \cite{Howe1998}.
%Development has been rapid, and this must be due, in no small part, to the common point of departure enabled by Lighthill.
%The relation of special cases to the whole is clear, it is easy to see the wood from the trees.
% The turbulent sources in Lighthill's theory will in general interact with each other.
% There is a long history in studying the interactions of vorticies \cite{Whittaker1951}, 
% but yet no 
% where the force between the rings `acts at a distance', and  propagates through the medium via sound waves.
% The vorticity formulation of Lighthill's equation, therefore,
% seems ideally placed to formulate sound-acoustic source dynamics in a general way.
% This is a difficult problem in general,
% but a simple solution is possible when the interaction is measured by ultrasound,
% and is outlined in \secref{Lorentz}.
% It turns out that ultrasound is simpler than acoustics in general due to the way in which ultrasound defines time and space.
% This is discussed more in \secref{Measurement}.

% In medical ultrasound, however, the interaction of purely turbulent sources of sound are not of primary interest.
% Rather, focus is directed at the  interactions between  micron-sized pulsating bubbles that are used as contrast agents.
% Usually, when modelling such interactions the fluid is linearised early \cite{Crum1974,Leighton1990, Mettin1997},
% and so any effect of vorticity and turbulence is ignored$rho.
% This is unsatisfactory for two reasons.
% \nlist{
% \item Although the pulsating bubble is likely to dominate the sound generated in a region of turbulence,
%   it does not necessarily follow that the {\em motion } of the bubble is independent of the vorticity in the surrounding fluid.
% \item By linearising the fluid the general framework for sound-acoustic source interactions is being abandoned.
% }
% The first of these problems is the greater when it comes to predicating the results of an experiment.
% However, the second is the more grave.
% By approximating too early the theory becomes separated from the general theory of sound-source interaction, and progress becomes slow.
% Indeed, the Bjerknes force law that is often used to describe the interactions of bubbles has changed little since it was written down in 1905 \cite{Bjerknes1905},
% and this is not because it is without  problems.

% In this report we attempt to bring the description of the motion of a bubble within the Lighthill framework for ultrasonics outlined in \secref{Lorentz}.
% This is attempted in \secref{Zitter}.






\subsection{The acoustic analogues to the electric and magnetic fields}\label{sec:int:EM}

In classical electromagnetism the electric and magnetic fields are 
3-dimensional vector fields that are measured (usually) in the laboratory frame.
Such spatial vector quantities we denote in bold.
%The laboratory frame is represented by an inertial and orthonormal frame with basis vectors $\{\gamma_\mu| \mu = 0,1,2,3\}$,
%where $\gamma_0$ is a timelike vector with positive signature ($\g^2 =1$) 
%and $\{\gamma_k | k=1,2,3\}$ are spacelike with negative signature ($\g^2=-1$).

The most direct method of obtaining the analogues  is to project the vorticity bivector, $F$, into the laboratory  frame\cite{Hestenes2003, Doran2003}.
The analogue to the electric field can then be defined to be the timelike component, and the analogue to the magnetic field the spacelike component.
The directness of this method, however, comes at the cost of it bearing little  resemblance to conventional acoustics.

To demonstrate the similarities and the differences of ultrasound formulation of acoustics  to the Gallilean formulation,
we re-derive Maxwell's relation using an argument very similar to Lighthill's acoustic analogy\cite{Lighthill1952} of aeroacoustics.
The analogues to the electric and magnetic field  become clear in this process.

We start  by projecting the temporal and spatial equations of motion, equations \eqnref{eomTime} and \eqnref{eomSpace}
into the laboratory frame.
The result is
\sub{
  \begin{align}
     \vdel \cdot \vA &=  - \dt \phi, \label{eqn:Rcontinuity}\\ % &\quad\text{and} &&
\dt \vA - \vv \times  \lr{\vdel \times \vA} \label{eqn:REuler}
&= - \vdel \phi.
  \end{align}
}
$\phi$ and $\vA$ are the temporal and spatial components of the vector potential $A$.
%\eqa{
%\phi &\equiv  \gamma \lr{c^2 + w}  \\ %= A \cdot \g,\gamma \frac{\epsilon + p}{\rho} 2\gamma  p^{1/2}
%& \quad\text{and} &&
%\vA &\equiv  \phi \vv,  %A \wedge \g.
%}
As we found, $\phi$ may be interpreted as the total enthalpy density,
and $\vA = \phi \vv$.
$\vv$ is the three dimensional velocity.

%Equation \eqnref{continuity} is the ultrasound equiv
Equations \eqnref{Rcontinuity} and \eqnref{REuler} are the acoustically measured versions of the continuity and Euler equations.
In the non-relativistic limit the equations reduce to Galilean invariant forms, 
\sub{
\begin{align}
  \vdel\cdot\lr{\rho \vv}  &= -\dt \rho, \label{eqn:NRcontinuity}\\
  \dt \vv - \vv\times \lr{\vdel\times \vv} &= - \vdel \phi,\label{eqn:NREuler}
\end{align}
}
with 
equation \eqnref{NREuler} being Euler's equation written in Crocco's form\cite{Howe1998}.
The difference between the two sets of equations is that the mass density, $\rho$, in the Gallilean forms is
replaced by the total enthalpy density $\phi$. 
Replacing the mass with an energy is typical of Lorentz invariant theories.

With the continuity and Euler equations in hand,
we may now apply the conventional formulations of acoustics.
We proceed with Lighthill's acoustic analogy\cite{Lighthill1952, Howe1998}.

To do so we   differentiate  the continuity equation (equation \eqnref{Rcontinuity}) with respect to time 
and subtract it from the spatial derivative of  Euler's equation (equation \eqnref{REuler}).
A wave equation for the  total enthalpy results
\subl{
\eqa{
   \lr{\vdel^2 - \dt^2}\phi
  & = \vdel \cdot \lr{\vv \times\lr{\vdel \times \vA} } \equiv - \rho_q.
\label{eqn:WavePhi}
  \intertext{Next, a wave equation for $\vA$ is obtained by 
    by differentiating the continuity equation with respect to space 
    and then adding the result to the temporal derivative of Euler's equation,
    }
   \lr{\vdel^2 - \dt^2}\vA 
   &  = - \vdel\times\lr{\vdel \times \vA} - \dt \lr{ \vv \times \lr{\vdel \times \vA}} \equiv -\vJ.
    \label{eqn:WavevA}
  }
}{Waves}
In keeping with Lighthill's analogy we interpret the right hand side of \eqnref{WavePhi} and  \eqnref{WavevA} 
as an acoustic source density, $\rho_q$, and acoustic current density, $\vJ$, 
respectively.
% Notice the nice property that the current is conserved $\vdel\cdot \vJ + \dt \rho_q = 0$.

For comparison, 
had we carried out this procedure with the Gallilean continuity and Euler equation we would have obtained\cite{Howe1998},
\subl{
  \begin{align}
    \lrsquare{  \Dt \lr{\frac{1}{c^2} \Dt} - \frac{1}{\rho}\vdel \cdot \lr{\rho \vdel}}\phi &= -\frac{1}{\rho} \vdel \cdot \lr{\rho \vv \times \lr{\vdel\times \vv}} \label{eqn:WavephiREG}\\
    \lrsquare{  \Dt \lr{\frac{1}{c^2} \Dt} - \frac{1}{\rho}\vdel \cdot \lr{\rho \vdel}}\vv  &= \frac{1}{\rho} \vdel \times \lr{\rho \vdel \times \vv}. \label{eqn:WavevvREG}
  \end{align}
}{WavesREG}
These are Lighthill's equations expressed in terms of enthalpy and vorticity\cite{Howe1998}.
The operator  $\Dt = \dt + \vv \cdot \vdel$.
The left hand side of both equations \eqnref{WavesREG} describe a non-linear wave in homoentropic potential flow \cite{Howe1998}.

Equations \eqnref{WavePhi} and \eqnref{WavevA} can be simplified by introducing
\begin{align}
  \vE &= - \vv \times \lr{ \vdel \times \vA} &\text{and}&&
  \vB &= \vdel \times \vA,
\end{align}
so that
\subl{
\eqa{
   \lr{\vdel^2 - \dt^2}\phi
  & = - \vdel \cdot \vE  \equiv - \rho_q.
\label{eqn:WavePhiMax}
  \intertext{and
    }
   \lr{\vdel^2 - \dt^2}\vA 
   &  = - \vdel\times\vB + \vE \equiv -\vJ.
    \label{eqn:WavevAMax}
  }
}{WavesMax}
Equations \eqnref{WavesMax} can now be recognised as Maxwell's equations written in terms of the potentials in the Lorenz gauge\cite{Doran2003}.
The vector $\vE$ is the Coriolis acceleration, and takes the role of the electric field in the analogy.
The axial vector $\vB$ is the spatial vorticity and takes the role of the magnetic field.

Writing out Maxwell's 4 equations explicitly gives
\sub{
\eqa{
 \vdel \cdot \vE &= q,\label{eqn:M1}  \\ 
 \vdel \times \vB &= \vJ + \dt E, \label{eqn:M2}\\
 \vdel \times \vE &= -\dt\vB\label{eqn:M3},\\
 \vdel \cdot \vB &= 0\label{eqn:M4},
}
}
The acoustic interpretation of these equations are as follows:
\nlist{
\item Equation \eqnref{M1} is the definition of an acoustic source.
\item Equation \eqnref{M2} is the definition of the acoustic current.
\item Equation \eqnref{M3} is the Lorentz invariant version of the vorticity equation.
\item Equation \eqnref{M4} is an expression of Helmholtz theorem, which demands the conservation of vorticity.
}

%Equations  \eqnref{Waves} are linear, as expected.
%However, it must be emphasised that the lack of non-linearity in \eqnref{Waves} 
%is not due to approximation -  
%equations \eqnref{Waves} and \eqnref{WavesREG} are equally exact.
%Ultrasound acoustics is simpler because less can be 
%known about the medium when it is measured acoustically.

%forms,where $\rho$ is the mass
%It is worthwhile here making a direct comparison.




% The 
% so that
% \eqa{
%  F = F\g\g = \lr{F\cdot\g}\g + \lr{F\wedge \g}\g.
% }
% The analogue to the electric field, $\vE$, and magnetic field, $\vB$,
% %In keeping with convensional formulations of classical electromagnetism,
% %we define the analogues to the electric field, $\vE$, and magnetic field, $\vB$,
% can then be defined \cite{Hestenes, Doran2003} to be 
% \sub{ 
% \label{eqn:DefnEB}
% \eqa{
%   \vE &\equiv  \lr{F\cdot\g}\g = \frac{1}{2}\lr{ F- \g F \g  } = -\dt \vA - \vdel \phi\label{eqn:DefnE}\\
%   \intertext{and}
%   I\vB  &\equiv  \lr{F\wedge\g}\g =  \frac{1}{2}\lr{ F+ \g F \g } = \vdel \wedge \vA,\label{eqn:DefnB}
% }
% }
% respectively.
% Here  $\phi = 2\gamma\sqrt{p}$
% and $\vA = \phi \vv$ are the temporal and spacial components of $A$,
% $\gamma = (1-\vv^2)^{-1/2}$ is the Lorentz  factor and $\vv$ is the spatial velocity vector.
% The term $I$ in \eqnref{DefnB} represents the spacetime pseudoscalar,
% or the `unit volume element' of spacetime.

% To find the rightmost side of \eqnref{DefnE} and \eqnref{DefnB} equation \eqnref{DefnVorticity} has been used,
% and the vector identity $a \cdot \lr{d B} = a\wedge B + d a\wedge B$ for vectors $a$ and $d$, and multivector $B$, is helpful for  \eqnref{DefnB}.
% The spacetime vorticity then exhibits the same `complex' structure as the electromagnetic field tensor,
% \eqal{
% F = \vE + I\vB,
% }{FSplit}
% where the volume $I$ takes the role of the complex number%]
% \footnote{
%   Note that $I^2= \gamma_0\gamma_1\gamma_2\gamma_3 \gamma_0\gamma_1\gamma_2\gamma_3 = -1$.
%   This geometric interpretation of complex numbers is typical of geometric algebra,
%   and gives the algebra huge power\cite{imaginary_numbers_are_not_real}.
% }.

% Expressing \eqnref{DefnB} in terms of the convensional Gibbs vector cross product, denoted with a $\times$,
% we find that $\vB = -I \vdel \wedge \vA \equiv \vdel \times \vA$, 
% and so that the axial vorticity vector is the analogue to the magnetic field.

% An alternative expression for the analogue to the electric field can be found by using equation \eqnref{FSplit} 
% in the orthogonality condition for the velocity and vorticitiy (equation \eqnref{eomSpace}).
% By considering only the bivector quantities (specified by the subscript 2 on the angular brackets $\bivector{\cdot}$)
% it is found that
% \eqa{
%   \bivector{ \g v\cdot F} = \half\bivector{\g v F - \g F v} =\gamma \vE - \gamma \vv \cdot \lr{\vdel\wedge\vA} = 0.
% }
% In the calculation it is helpful to note that $I\vB$ commutes with $\g$ and $\vE$ anti-commutes with $\g$,
% as follows from the second equalities in \eqnref{DefnEB}.
% Therefore, 
% \eqa{
%   \vE = \vv \cdot \lr{\vdel \wedge \vA} \equiv - \vv \times \lr{\vdel \times \vA}, \label{eqn:E}
% }
% which is the Coriolis acceleration.
% The second identity is the result writen with the familiar Gibbs vector product.

% %To prove that the two equations \eqnref{vE1} and \eqnref{vE2} are consistent we note that
% %\eqa{
% %\vv \cdot \lr{\vdel \wedge \vA} = \vv \cdot \vdel \vA - \vdel \vv \cdot \vA = \dt \vA - 
% %}

% \subsection{Maxwell's relation in terms of the Coriolis acceleration and the vorticity}

% Taking the spacetime-split of \eqnref{Maxwell} and using
% \eqnref{FSplit}
% we obtain
% \eql{
%   \lr{\dt + \vdel} \lr{\vE + I \vB} = \rho - \vJ
% }{MaxwellSTSplit}
% where
% \eqa{
% q = J \cdot \g \quad\text{and} \quad
% \vJ  = J\wedge \g.
% }
% Separating out the respective scalar, relative vector,
% relative bivector and relative trivector parts of \eqnref{MaxwellSTSplit}  returns  four equations,
% % \eqa{
% %   \vdel \cdot \vE &= \rho,\\
% %   \dt \vE + \vdel\cdot (I\vB)  &= -\vJ,\\
% %   \del\wedge\vE + \dt(I\vB) &= 0,\\
% %   \del\wedge\vB &= 0.
% % }
% % These may be rewritten 
% \sub{
% \eqa{
%  \vdel \cdot \vE &= q,\label{eqn:M1}  \\ 
%  \vdel \wedge \vB &= I\lr{\vJ + c^2\dt E}, \label{eqn:M2}\\
%  \vdel \wedge \vE &= -\dt(I\vB),\\
%  \vdel \cdot \vB &= 0,
% }
% }
% which are Maxwell Equations written explicitly in terms of the
% acoustic fields.


\subsubsection{The acoustic sources}
% The acoustic
% From Maxwell's equations we may write explicit forms for the 
% acoustic sources.
% From \eqnref{M1} and \eqnref{E} we have
% \sub{
% \begin{align}
%   q = -\vdel \cdot\lr{ \vv \times \lr{\vdel \times \vA}} \label{eqn:rho}
% \end{align}
% and from equations \eqnref{M2}, \eqnref{E} and \eqnref{DefnB}  we have
% \begin{align}
% \vJ = \vdel \times \lr{\vdel\times \vA} + \dt \lr{ \vv \times \vdel \times \vA}.\label{eqn:vJ}
% \end{align}
% \label{eqn:sources}
% }
% %Note that reintroducing SI units forces the
% %analogue to the electric permittivity to be unity
% %and the analogue to the magnetic permeability to be $c^{-2}$.

% In the non-relativistic limit, equation \eqnref{rho} is 
% the turbulence source term of  Lighthill's formulation of aeroacoustics,
% when written in terms of the total enthalpy and the vorticity\cite{Howe1998}.
% Equation \eqnref{vJ}, however,  seems seldom to be used in the aeroacoustics literature.


It is interesting to note the dimensionality of the acoustic source.
To do so it is necessary to reintroduce SI units so that \eqnref{M1} and \eqnref{M2} become
\sub{
\begin{align}
  \frac{\rho_q}{\epsilon_0} = -\vdel \cdot\lr{ \vv/c \times \lr{\vdel \times \vA}} \label{eqn:rho}
%  \lrsquare{q} = \frac{\text{Energy}}{\text{distance}^2} = \frac{\text{mass}}{\text{time}^2}.
\end{align}
and 
\begin{align}
 \frac{\vJ}{c^2 \epsilon_0} = \vdel \times \lr{\vdel\times \frac{\vA}{c}} + \frac{1}{c^2}\dt \lr{ \vv \times \vdel \times \vA}.\label{eqn:vJ}
\end{align}
}
where $\epsilon_0$ is the acoustic analogue to the electric permittivity.
The units of the left hand side of \eqnref{rho} are
\begin{align}
  \lrsquare{\frac{\rho_q}{\epsilon_0}} = \frac{\text{Energy}}{\text{distance}^2}.
\end{align}
The analogue units in electromagnetism are 
\begin{align}
\lrsquare{\given { \frac{\rho_q}{\epsilon_0}}{\text{em}}} =  \frac{\text{Energy}}{\text{distance}^2\times \text{charge}}.
\end{align}
It follows that the acoustic analogue to charge is {\em dimensionless},
and the analogue to permittivity has the units 
\begin{align}
\lrsquare{\epsilon_0} = \frac{1}{\text{distance}\times\text{Energy}}.
\end{align}
We do not, however, have an adequate interpretation of the acoustic permittivity.
We suppress the issue by  returning to c.g.s. units, where $c= \epsilon_0 = 1$.


%This emphasises what is already obvious from \eqnref{rho},
%that the acoustic charge is a dynamic property of the flow.
%The unit of $\vJ$ is a source term multiplied by a velocity.
%It is consistent with  the current  being a moving charge,
%\begin{align}
%  \vJ = q \vu
%\end{align}
%with $\vu$ being the velocity of the source.


\subsection{Changes in Gauge}


The electromagnetic gauge transformation,
\begin{align}
  A^\prime = A - \del \psi,
\end{align}
in the acoustic analogy corresponds to an introduction of a potential flow of the medium.
To demonstrate this, we first note that the relativistic generalisation to the velocity potential, $\psi$,  is defined by\cite{LandauBook}
\begin{align}
  \del \psi = - \frac{\epsilon+p}{n} v = - \frac{2\epsilon}{n} v,
\end{align}
where $n$ is the proper particle number density of the fluid and equation \eqnref{eos} has been used for the second equality.
To show that this is equal to the negative of the potential $A$, we use an argument of Taub\cite{Taub1978}.
The internal energy, $\epsilon$, is equal to the sum of the rest mass and the thermodynamic internal energy \cite{LandauBook, Doran2003},
\begin{align}
  \epsilon(p) = nm( 1 + e(p)),
\end{align}
where $m$ is the proper mass.
From the isentropic thermodynamic relation $m de = - p d\lr{\frac{1}{n}}$
it follows that 
\begin{align}
 n d\epsilon = \epsilon dn - n^2 p d \lr{\frac{1}{n}} = \lr{\epsilon + p} dn.
\end{align}
Applying equation \eqnref{eos} and integrating we obtain
\begin{align}
n = \sqrt p,
\end{align}
from which we find
\begin{align}
A = 2\sqrt p  v = \frac{2\epsilon}{n} v = - \del \psi.
\end{align}
Therefore, and as asserted, the presence of a potential flow in the acoustic medium corresponds to a gauge transformation of Maxwell's equations,
and hence, the description of the acoustic wave is unaffected by potential flows of the medium.


\subsection{The interpretation of a sound pulse}\label{sec:Interpretations}

The equivalence of the formulation of ultrasound measured acoustics and electromagnetism raises a number of issues of interpretation.
The first of these is the linearity of the theory.
Nonlinear propagation of the sound pulse is well known to be important in ultrasound physics, 
and yet it vanishes altogether in our formulation.
This issue is considered in \secref{NonlinearProp}.

The second is that a sound pulse is a longitudinal wave, the compressions and rarefactions of the density are in the direction of propagation,
whereas Maxwell's relations describe a pair of self interacting transverse waves (the Coriolis acceleration and the vorticity).
The compatibility of these two equations is discussed in \secref{transversivity}

\subsubsection{On the absence of non-linear propagation} \label{sec:NonlinearProp}

The linearity of the acoustic field is entirely appropriate for ultrasound.
To apply a non-linear wave equation you need to know 
the spatial variations of the speed of sound. 
%which depend in turn, upon the spatial variations of the density.
This is seen in the left hand side of equations \eqnref{WavesREG}.
Such knowledge is impossible
when using sound to define the concept of distance:
the sound speed used by ultrasound must be a spatially invariant constant to be able to measure anything at all.
What is to be explained, therefore, 
is not why ultrasound measured acoustics is linear - this is surely correct - 
but to explain how the non-linearity found in other measurement systems  manifests itself in acoustic measurements.
To do so it is useful to frame the discussion around the linear acoustically measured equations of
\eqnref{Waves}
and their non-linear Galilean forms \eqnref{WavesREG}.

The first point to note is that the non-linearity of the Galilean formulation of sound is entirely a matter of {\em  convention}.
It would in fact be more appropriate to rewrite equations \eqnref{WavesREG} as linear wave equations with everything else interpreted as acoustic sources and currents.
Then the sound is defined as the part of an acoustic disturbance that can propagate energy away to infinity,
the rest of the disturbance being a `local' source.
This is, in fact, the usual final step of Lighthill's analogy.
The reason it is rarely performed when the analogy is written in terms of the total enthalpy and vorticity is because
the acoustic source terms become horribly complicated.
The split of source and wave in \eqnref{WavesREG} is convenient  interpretatively, but is nevertheless rather ad-hoc,
for it mixes local terms with those that can propagate indefinitely.

The influence of non-linear propagation on what can be measured acoustically is found by comparing the right hand sides of equations \eqnref{Waves} and equations \eqnref{WavesREG}.
It is seen that the only major difference between the two is the term $- \dt \lr{ \vv \times \lr{\vdel \times \vA}}$ on the right-hand-side of \eqnref{WavevA}.
This term is, if you like, the ghost of the non-linear operator $\Dt$ on what can be measured acoustically.
When measured with ultrasound it is interpreted  as part of the current.

We note that an attempt to re-incorporate the ghost  term back into some `acoustically measured non-linear operator' would be ill-conceived, 
for it would mean that the acoustic current is no longer conserved:
the term $- \dt \lr{ \vv \times \lr{\vdel \times \vA}}$ most certainly is a current.






\subsubsection{Interpreting the transversivity of the sound pulse} \label{sec:transversivity}


An acoustic plain wave is  a  perturbation of the fluid particles in the direction of propagation of the pulse.
It is also a transverse wave in terms of the Coriolis and vorticity fields.
We now propose a method of squaring these seemingly contradictory views.

We consider a segment of the plain wave to be a narrow tube,
so that the entire plain wave is formed by adding together many such tubes.
Within the tube a pertabation in pressure propagates the sound wave longitudinally.
Outside of the tube no such pertabation exists.

Both within and outside of the tube  there is no vorticity and so there is no propagation of the transverse wave.
On the interface of the tube, however, there is shearing of the fluid:
the molecules  within tube move as the wave passes, the molecules outside do not.
This shearing induces a {\em vortex sheet} on the interface \cite{Howe1998}.
The vorticity is confined to the sheet 
and has a strength equal to the average speed of the molecules on either side of the interface.
From the acoustic analogue Maxwell's equations, this vorticity then induces a Coriolis acceleration orthogonal to the interface.
A longitudinal wave therefore  induces a transverse wave on its boundary.

There remains a difficulty with this interpretation, however.
A transverse wave should have exactly two helicity states.
The interpretation given to the first seems, to the author at least, entirely plausible.
The second is more difficult, however.
The construction of a transverse wave with the Coriolis and vorticity vectors interchanged does not seem obvious.
Rather than speculate, 
we leave this question unanswered.



% for the direction of the vortex  difficult, however. , We have given an 
% That is, the vorticity and the 


% Within the perturbation the vorticity vanishes
% and so it is temping to state that $\vB = 0$, innihilating the helicity with it.
% However, this view is not correct.
% To see this we imagine a  wave confined to a narrow tube of radius $r$.
% The plane wave can then be constructed from many  such tubes (see \figref{tubes}).

% At the boundary of an isolated tube slipping occurs between the translating particles that carry the sound
% within the tube
% and the bulk fluid outside.
% This results in a {\em vortex-sheet}\cite{Howe} on the boundary of the tube.
% Also at the boundary there is a discontinuity in the enthalpy $\phi$, 
% carried over from the velocity contribution of $\gamma$ in \eqnref{}.
% The Corriolis acceleration, from  \eqnref{}, therefore has two components.
% A component parallel to the tube $\vE_\parallel$ from the $\dot A$ term and a perpendicular component $\vE_\perp$ from the $\vdel \phi$ term.


% which contains 
% The vorticity from adjacent tubes cancel, 
% but the vorticity at the boundary never does.
% From equation \eqnref{} (Crocco's equation, in the hydrodynamics literature)
% The vorticity on the boundary induces a Coriollis acceleration,
% we see that on the boundary the vorticity 

 
% and it is this sense that a sound pulse can be considered as a transverse Coriollis-voriticity wave.
% The transversivity must can be set in two directions which can be described as being in either a {\em left-}  or {\em right-handed polarisation state},
% and so the helicity must be in one of two states.


% Nevertheless, for the sound pulse not to have a preferred direction, 
% and therefore to conserve angular momentum,
% it is required that it the particle  streamlines form a knot over this infinite interval.
% Perhaps the simplest vortex  knot that may be considered is the trefoil knot illustrated in \figref{}.
% Topologically the trefoil knot is equivalent to two linked rings,
% each  with the same circulation, $\kappa$, as the knot.
% The helicity of the trefoil knot is accordingly $\pm2\kappa^2$.


% The helicity of a photon is $\pm\hbar$.
% If sound is knotted with the simple trefoil knot then the {\em acoustic Plank's constant} would be
% \begin{align}
%   \hbar = 2\kappa^2,
% \end{align}
% (divided by a unit momentum, see footnote \ref{footnote:dimensionallity_footnote}).

% Why $\hbar$ and therefore $\kappa$


% Moffatt's topological interpretation of helicity has been used before to  interpret the quantisation
% of the spin in electrodynamics.
% See, for example the extensive investigations into the {\em magnetic helicity} of Trueba and Ra{\~n}ada\cite{Trueba1996, Trueba2000, Ranada2002}
% (The magnetic helicity is equation \eqnref{Helicity} where the $\vA$ and $\vB$ have there electromagnetic interpretation).
% %This in fact continues the very old notion of the vortex atom
% %that goes back to the very formulation of electrodynamics.
% What is new
% in this thesis is the equivalence of electromagnetic field and the acoustical field.
% The hydrodynamic helicity is the {\em same} as the acoustic helicity;
% the  sound pulse has a quantised spin
% and  this spin may be interpreted, via the  helicity,
% as the orbital angular momentum of  the fluid particles about their mean flow.
% %\cite{Moffatt1969, Moffatt1988, Chechkin1993,  Trueba1996, Trueba2000}.

\subsection{Discussion}

This  chapter has explicitly formulated the (exact) analogy between ultrasound measured acoustics 
and electromagnetism.
It has been found that the acoustic analogue to the electric field is the Coriolis acceleration,
and that the acoustic analogue to the magnetic field is the vorticity,
The spacetime vorticity bivector takes the role of the field tensor.
The analogy in this form has long been suspected\cite{Marmanis2000,Sridhar1998},
however, to the authors knowledge this is the first time that the analogy has been  completed.
The key step, missing in previous attempts, 
is to note that acoustics must be formulated in terms of a Lorentz invariant fluid where
{\em the speed of sound equals the speed of light}.
It is only when this step is made that the analogy exists.

Relativistic fluids where the sound speed equals the speed of light have been studied many times before
as theoretical curiosities\cite{Taub1978,Pekeris1977}.
For example, Pekeris observed that Hick's spherical vortex conserves angular momentum if and only if
the sound speed equals the speed of light\cite{Pekeris1977}.
The importance of such fluids, however, has not to the authors knowledge been recognised.
Such fluids represent {\em what can be measured} when distances are obtained by echo-location.
Acoustics as measured with ultrasound is therefore {\em identical} to electromagnetism (as measured with light).
Retrospectively this correspondence is not too surprising.
For both acoustics as measured with ultrasound and electromagnetism as measured with light 
attempt to measure the properties of their propagating signal.
Both, therefore, 
represent a similarly limited view of the world,
the limitations manifesting themselves in the linearity of the equations.

Another interesting analogy between acoustics and electromagnetism is `acoustic analogue gravity' literature (see Barcel{\'o}, Liberati and Visser\cite{Barcelo2005} for a review).
The approach constructs an {\em acoustic} metric that describes the acoustics of sound carried in bulk flow.
While the description of space and time in this formulation is Euclidean, the acoustic metric turns out to be pseudo-Euclidean,
and therefore obeys the Lorentz transformation.
This results because sound carried away in bulk flow  faster than the speed of sound will never reach us.
The speed of sound is  therefore a limiting velocity in transformations.
The analogue gravity literature then goes on to study the gravitational implications of the acoustic metric.

While the motivations behind the two approaches is similar, in particular the demand that acoustics must be obey a Lorentz invariant metric,
the approach given here and the approach of analogue gravity are fundamentally different.
Analogue gravity does not consider the measurement process and so operates within a world characterised by two metrics, 
the Lorentz invariant acoustic metric and the Galilean invariant spacetime  metric.
Ultrasound measurement demands that the world be described by a Lorentz invariant metric.
There is no other way to describe space and time acoustically.
In analogue gravity the acoustic metric is Lorentz-invariant, but is not the same as the metric used here.
In analogue gravity the metric is a function of the bulk flow,
whereas  we argue that this is impossible:
the sound speed must be an a priori constant in order to say anything about the world.
Nevertheless, the success of analogue gravity is very encouraging.
Acoustically observed microbubbles could offer an interesting experimental model in this field.



% The speed of sound is similarly limiting.
% This observation is then used to pursue the 
% The approach  starts by noting that sound carried away by a bulk flow travelling faster than the speed of sound will never reach us.
% In this sense, the acoustic source is silent, and not, perhaps dissimilar to an a
% emitted by an acoustic source that is carried away by bulk flow faster than the speed of sound.Starting from the observation that it is impossible to measure the acoustics of a 

\subsubsection{The acoustic Lagrangian}\label{sec:int:spin}
Finally, let us use the acoustic analogy to set out some directions for future work.
%%
%
%Usually, when imagining the average motion of particles in a sound wave,
%we think of the particles oscillating back and forth as the pressure perturbation passes through.
%This is because we think of sound waves as being longitudinal.
%In equation \eqnref{Maxwell} we have an exact analogy between electromagnetism and ultrasound measured acoustics.
%that a sound waves can be considered to be a self-inducing vorticity-Coriolis wave.
%The transverse nature (with respect to the direction of propagation)  of electromagnetic waves is well known, and applies equally to the vorticity and Coriolis fields here.
%In this section we consider the implications of this observation.
%\subsection{The acoustic Lagrangian}\label{sec:lagrangian}
%From equations \eqnref{Maxwell} acoustic sources are described by the same equations that characterise electromagnetic sources.
The common mathematical  description between acoustics and electromagnetism can be enforced by deriving both from a Lagrangian of the same form.
The electromagnetic Lagrangian density is\cite{Lasenby1993, Doran2003},
\begin{align}
 \L = \frac{1}{2} F \cdot F  - A \cdot J, \label{eqn:Lagrangian}
\end{align}
and the acoustic Lagrangian is same so long as the symbols $F$, $A$ and $J$ take their 
acoustical meaning.
Note that the acoustical Lagrangian is not the same as the Lagrangian of the ideal fluid.
It describes directly the sound and the acoustic sources,
rather than the fundamental motions of the fluid particles.



%\subsection{Symmetries from the Lagrangian density}
The symmetries of \eqnref{Lagrangian} can be found by minimising the action,
\begin{align}
  S =  \int \abs{d^4x} \L(A, \del\wedge A; x).
\end{align}
In the absence of acoustic sources ($J=0$) this results in the Euler-Lagrange equation 
\begin{align}
  \d_A \L + \del \cdot  \lr{\del_{\del\wedge A}\L} = 0. \label{eqn:EL}
\end{align}
Demanding translational invariance of the action and assuming that the
Euler-Lagrange equations are satisfied results (via Noether's theorem) in the conservation of the acoustic energy momentum tensor\cite{Lasenby1993,Doran2003}  
\eqa{
   T\lr{a} = -\half F a F.
} 
Demanding rotational invariance and applying \eqnref{EL} results in the conservation of the angular momentum.
The adjoint to the angular momentum tensor is the most useful and is\cite{Lasenby1993,Doran2003}  
\begin{align}
 \adjoint{\J}(n) = A \wedge \lr{F\cdot n} +  \adjoint{T}(n) \wedge x.
\end{align}
The second term,  $\adjoint{T}(n) \wedge x$, is the orbital angular momentum of the sound pulse as
moves through space.
The first,
\begin{align}
S(n) = A \wedge \lr{F\cdot n},
\end{align}
is the intrinsic spin of the  acoustic field.

The acoustic energy momentum tensor can be applied immediately to derive the 
 force law for an acoustic source when in the presence of an external vorticity or Coriolis field.
Remembering Maxwell's relation, $\del\cdot F = J$ (equation \eqnref{Maxwell})
we obtain 
\begin{align}
 f = \scope T(\scope \del) = -\half \lr{\scope F\scope \del F + F \del F} = -\half \lr{-JF +FJ} = J \cdot F.
\end{align}
If the acoustic source has a constant rest mass, $m_0$, and moves at a speed $u$,
then, by writing  $J = q u$, where $q$ is the acoustic source, we obtain
\begin{align}
  f = m_0 \dot u = q u \cdot F, \label{eqn:LFL}
\end{align}
Equation \eqnref{LFL}  is the Lorentz force law.

To write equation \eqnref{LFL} in its more conventional form we project into the laboratory frame,
\begin{align}
J\cdot F  = \scalar{\lr{\rho + \vJ} \g\lr{\vE+I\vB}} = -\lr{J\cdot E + q \vE + q \vu\times\vB }\g.
\end{align}
The timelike $q\vu\cdot\vE $ is the work done.
The remaining spatial component is
\begin{align}
   \vf = -q \lr{\vE + \vu \times \vB},
\end{align}
which is the Lorentz force law in its usual form.

Equation \eqnref{LFL} describes how a turbulent source moves 
in the presence of external flows when measured with ultrasound.
It is an incredibly powerful result.
Its use, however, requires individual acoustic sources to be tracked though space.
This is generally beyond the capability of ultrasound,
although this is changing with the increasing availability of 3 dimensional probes.

%Its linearity is an example of the general rule that the world appears more simple
%when it is measured acoustically.
%This is because, due to the limitations on what ultrasound can measure, reassigns the temporal and spatial locations of entities so that 
%the non-linearity in their motion vanishes.

\subsubsection{Helicity and Spin}

The existence of the intrinsic acoustic spin raises further interpretative issues.
In the absence of satisfactory explanation for these, however, we resist the temptation to speculate.
All we say on the matter is to note that helicity of the sound wave, 
the projection of the spin on the direction of the momentum, 
is identical to the hydro dynamical helicity introduced by Moffatt\cite{Moffatt1969}.
This is interesting because the integral of the hydrodynamic helicity is a topological invariant that measures the degree of knottedness of the flow\cite{Moffatt1969}.

To see this, we introduce the helicity,
\begin{align}
  \H \equiv   \hat \vP \cdot S(\gamma_0) = \hat \vP \cdot \lr{\phi \vE - \vA \times\vE} =  \frac{\abs{\vE}^2}{\abs{\vE\times\vB}}\vA \cdot \vB.
\end{align}
where  $\hat \vP$ is the unit  Poynting vector, 
\begin{align}
 \hat\vP =\frac{\vE \times \vB}{\abs{\vE\times\vB}}.
\end{align}
and $S(\gamma_0)$ is the timelike component of the Spin 
as measured in the laboratory frame,
\begin{align}
 S(\gamma_0) % &=  A \wedge(F\cdot)  \\
 &= \half\bivector{A\g \g \lr{F\g - \g F}}\\
 &= \phi\vE - \vA \times \vE.
\end{align}
For a sound wave
%\begin{align}
$\abs {\vE} = c\abs{\vB}$
%\end{align}
and so the helicity simplifies to
\begin{align}
  \label{eqn:Helicity}
\H = \vA \cdot \vB.
\end{align}

% To interpret the spin it is useful to project it onto the direction of motion,
% yielding a scalar known as the helicity,
% \begin{align}
%   \H = \hat \vP \cdot S(\gamma_0),
% \end{align}
% where  $\hat \vP$ is the unit  Poynting vector, 
% \begin{align}
%  \hat\vP =\frac{\vE \times \vB}{\abs{\vE\times\vB}}.
% \end{align}
% and $S(\gamma_0)$ is the timelike component of the Spin 
% as measured in the laboratory frame,
% \begin{align}
%  S(\gamma_0) % &=  A \wedge(F\cdot)  \\
%  &= \half\bivector{A\g \g \lr{F\g - \g F}}\\
%  &= \phi\vE - \vA \times \vE.
% \end{align}
% The helicity is therefore
% \begin{align}
%   \H =  \hat \vP \cdot \lr{\phi \vE - \vA \times\vE} =  \frac{\abs{\vE}^2}{\abs{\vE\times\vB}}\vA \cdot \vB.
% \end{align}
% For a sound wave
% %\begin{align}
% $\abs {\vE} = c\abs{\vB}$
% %\end{align}
% and so the helicity simplifies to
% \begin{align}
%   \label{eqn:Helicity}
% \H = \vA \cdot \vB.
% \end{align}
% While $\H$ in \eqnref{Helicity} is defined in terms of the spin,
%  the 
 The term $\vA \cdot \vB$ on the right hand side is the same as the 
relativistic generalisation\footnote{ \label{footnote:dimensionallity_footnote}
The relativistic generalisation is accomplished by replacing $\vv$ in Moffatt's definition with $\vA$.
Notice, furthermore, that Moffatt did not normalise the Helicity,
that is $\H_{\text{Moffatt}} = p\cdot S$ rather than $\frac{1}{\abs{p}}p\cdot S$,
where $p$ is the momentum.
When we make the comparison to $\H = \epsilon_0 A\cdot B$, therefore,
we implicitly divide $\H_{\text{Moffatt}}$ by a unit momentum,
so that dimensionally all is correct.
} to the
{\em hydrodynamic helicity} per unit volume defined by Moffatt\cite{Moffatt1969}.
It is also  readily interpretable.
$\vB = \vdel \times \vA$ measures a rotation,
and the rotation is projected about  $\vA$.
The helicity therefore measures the degree to which the fluid streamlines rotate about themselves,
that is, the degree to which the streamlines are helical\cite{Moffatt1969, Ranada1992}.
%In a small volume $dV$ the flow is comprised of a superposition of a uniform flow $A_0$, a shear
%and a rigid body rotation about the origin of $dV$ with an angular velocity of twice the vorticity $2\vB_0$
%The streamlines of the flow within the sound pulse will be helices about the sheared $A_0$
The contribution to $\vA \cdot \vB dV \approx \vA_0\cdot \vB_0 dV$
is positive or negative depending on the orientation of the helix\cite{Moffatt1969}.

In this way Moffatt identifies the component of the spin parallel to the momentum
with the {\em orbital angular momentum} of a fluid streamline about its mean trajectory.
The equivalence between the two helicities (acoustic and hydrodynamic)
suggests that this interpretation of the spin is valid.

Finally we note that the hydrodynamic helicity was introduced by Moffatt because 
\begin{align}
  I = \int \abs{dV} \H = \int \abs{dV} \vA \cdot \vB
\end{align}
is an invariant that defines the degree of knottness of the fluid.
In the case of two vortex rings, for example, $\H =2 \alpha \kappa_1 \kappa_2$ 
where $\kappa_i$ are the circulations around the two rings. % $C_i$ of the two vortex filaments,
%\begin{align}
%  \kappa_i = \oint_{C_i} \vu \cdot d\vect{l},
%\end{align}
%where $\vu$ is the velocity around $C_i$,
%and $\alpha$ is an integer that defines the linking of the two filaments.
If they are unlinked $\alpha = 0$, whereas if they are singularly linked $\alpha = \pm 1$.
The helicity in a volume, the component of the spin parallel to the momentum, is therefore quantised
in accordance with the topology of the streamlines.

The author does not have an interpretation for this fascinating result and so does not comment further.
We note, however, that there has been considerable effort in applying the hydrodynamic helicity electromagnetism,
with implications to both the quantisation of the spin and to the quantisation of the charge.
We refer the interested reader to the literature\cite{Trueba1996, Trueba2000, Ranada2002}.




% that the projection of the spin along the direction of the acoustic momentum,
% We do not wish the resThis is necessary to avoid confusing the results already given 

% in the absence of sources ($J= 0$) by writing $A \rightarrow A + \epsilon \AA$,
% where $A$ is the  minimum field, $\epsilon$ is a small number and $\AA$ is a perturbation about $A$.
% Then 
% \begin{align}
% \frac{d S }{d\epsilon}&= \int \abs{d^4x}\lr{\AA \cdot \d_A \L + (\del\wedge \AA) \cdot \del_{\del\wedge A}\L}\\
% &= \int \abs{d^4x} \int \abs{d^4x} \AA \cdot\lr{ \d_A \L + \del \cdot  \lr{\del_{\del\wedge A}\L} } + \text{b.c}
% \end{align}
% where $\text{b.c}$ indicates the presence of  a total divergence which is assumed to vanish on the boundary.
% When $S$ is minimised,
% \begin{align}
%   \d_A \L + \del \cdot  \lr{\del_{\del\wedge A}\L} = 0, \label{eqn:EL}
% \end{align}
% which is the Euler-Lagrange equation for the acoustic field.
% x\subsubsection{Transformations}
% A transformation in the underlying space 
% \begin{align}
% x \rightarrow x^\prime = f(x)
% \end{align}
% maps one location $x$ in space to another, $\xp$.
% A tangent vector $a(x)$ to the space at $x$ 
% is then mapped to 
% \begin{align}
%    a(x^\prime) = a\cdot \del f(x) \equiv \fbar(a).
% \end{align}
% The adjoint, $\fadj$ to this transformation is defined by the relation
% \begin{align}
%    b^\prime \cdot \fbar(a) = a\cdot \fadj(b^\prime) 
% \end{align}
% so that 
% \begin{align}
% \fadj(b) = \del_a  \fbar(a)\cdot b  = \del f(x) \cdot b.
% \end{align}
% It follows that
% \begin{align}
%   \del a(\xp) &= \del_b b\cdot \del a(f(x))\\
%   %&= \del_b \lr{b\cdot \del f(x)} \cdot \del_{x^\prime} a(x^\prime) \\
%   &= \del_b \fbar(b) \cdot \del_{\xp} a(\xp)\\
%   &= \fadj(\del_\xp) a(\xp).
% \end{align}
% Therefore 
% \begin{align}
%   \del &= \fadj(\del_{x^\prime}) \equiv \fadj(\del^\prime). \label{eqn:delTransformation}
% \end{align}
% All vectors must transform the same way,
% and so more generally we define
% \begin{align}
%   a^\prime(x) = \fadj(a(x^\prime)) = \fadj\fbar(a(x)).
% \end{align}

% % In summary, therefore, the transformation $\xp =  f(x)$
% % induces the transformations
% % \sub{
% % \begin{align}
% %   a(x^\prime) &=  \fbar(a(x)) 
% % \intertext{and }
% %   \del &= \fadj(\del_{x^\prime}) \equiv \fadj(\del^\prime).
% % \end{align}
% % \label{eqn:InducedTransformations}
% % }
% The action transforms under a change of coordinates as 
% \begin{align}
%   S = \int \abs{d^4x} \L(A, \del\wedge A; x) = \int \abs{d^4x^\prime}\det \fbar^{-1} \L(A^\prime, \del^\prime\wedge A^\prime; x^\prime)
% \end{align}
% and so 
% \eqa{
%   \L^\prime(A, \del\wedge A;x) = \frac{1}{\det \fbar} \L(A^\prime, \del^\prime\wedge A^\prime; x^\prime).
% }
% Parameterising the transformation by the scalar $\alpha$,  where $\alpha=0$ corresponds to the identity transformation,
% gives
% \begin{align}
% \given{\frac{d \L^\prime }{d\alpha}}{\alpha=0} &= \delta A^\prime \cdot \del_{A^\prime}\L + \lr{\del^\prime\wedge (\delta A^\prime)}\cdot \del_{\del^\prime\wedge A^\prime} \L
% \end{align}
% where the shorthand $\delta A^\prime = \frac{d A^\prime}{d\alpha}$ has been used.
% If it is assumed that  $A$ satisfies the Euler-Lagrange equation,  \eqnref{EL}, 
% then 
% \begin{align}
%   \given{\frac{d \L^\prime }{d\alpha}}{\alpha=0}  &= \del\cdot\lr{\lr{\delta A^\prime} \cdot \del_{\del^\prime\wedge A^\prime} \L}.
%   \label{eqn:dL}
% \end{align}
% %on the condition that, on the second line,  $A$ satisfies the Euler-Lagrange equation,  \eqnref{EL}.
% %If the Lagrangian density is symmetrical to the transformation the $\L$ is independent of $\alpha$ and 
% %{\em conjuagate current} $\lr{\delta A} \cdot \del_{\del_{x^\prime}\wedge A^\prime} \L$ is conserved.

% \subsection{Translational Invariance}
% If there is no priviliged position in space then the Lagrangian  density should be unchanged by the transformation
% \begin{align}
%   x^\prime = x + \alpha a,
% \end{align}
% for constant vector $a$ and scalar $\alpha$.
% The differential to the transformation is $\fbar(a) = a$
% the adjoint is  $\fadj(a^\prime) = a^\prime$,
% and the determinant is 1.
% Therefore  $A^\prime(x) = A(x)$ and the $\del = \del^\prime$.
% We also have  $\delta A^\prime = a\cdot \del A$
% and  $\frac{d \L(x^\prime }{d\alpha} = a \cdot \del \L(x)$.

% Equation \eqnref{dL} then becomes 
% \begin{align}
%   a\cdot \del \L =  \lr{a\cdot \del A} \cdot \del_F \L = \lr{a\cdot \del A} \cdot F,
% \end{align}
% from which it follows that, 
% \begin{align}
% \del\cdot\lr{\lr{a\cdot \del A} \cdot F - a\L} = 0.
% \end{align}
% The energy-momentum tensor, 
% \begin{align}
%   T(a) = \lr{a\cdot \del A} \cdot F - a\L = \lr{a\cdot \del A} \cdot F -\half a F\cdot F, \label{TEM}
% \end{align}
% is therefore conserved for translations through space-time.

% The explicit dependence on $A$ can be moved to the bounding surface\cite{Lasenby1993,Doran2003}
% by writing  $a\cdot\lr{\del \wedge A} =a \cdot \del A -  \del\lr{ A \cdot a}$
% so that
% \begin{align}
% \label{eqn:TAEM}
%   \Tgi(a) = \lr{a\cdot F}\cdot F  -\half a F\cdot F = - \half F a F
% \end{align}
% plus a total divergence, which is assumed to vanish on the boundary at infinity.
% The energy momentum tensor is then in a  manifestly gauge invariant form within the volume.
% $\Tgi$ is equal to its adjoint $\bar {\Tgi}$
% and so translational symmetry implies that  $\del \cdot \Tgi(a) = a\cdot\scope\Tgi(\scope\del) = 0$.
% This follows because
% \eqa{
%   \bar T(a) &= -\half \del_b \scalar{a F b F} \\
%   &=-\half \del_b \scalar{b F a F} \\
%   &= -\half F a F = T(a).
% }
% \subsection{Acoustic spin from rotational invariance}

% In addition to translational invariance, 
% all physical theories should be invariant to rotations:
% there should be no preferred orientation to our experiments.

% Rotations are represented by the transformation 
% \begin{align}
%   x^\prime = \Rt x R, \label{eqn:RotTrans}
% \end{align}
% where $R = e^{\alpha B/2}$ and $\Rt = e^{-\alpha B/2}$
% each represent rotations by the angle $\alpha/2$ through the plain $B$%
% \footnote{
% If $x$ is in the plain $B$ then, $B$ anti-commutes with $x$
% so that $\Rt x R = x e^{\alpha B}$. 
% Furthermore, if $B$ is composed of normal vectors of the same signature,
% for example $B = \gamma_1\wedge\gamma_2 = \gamma_1\gamma_2$ then
% $B^2= -1$.
% Equation \eqnref{RotTrans} then comprises of a rotation in the `complex plain'  $\gamma_1\gamma_2$.
% For the three spatial rotations there are three such plains and so three such complex numbers.
% These were introduced by Hamilton as $i$, $j$, $k$ in his Quaternian algebra,
% and is where, historically, the index notation comes for the three spacial axes.
% The double sided rotation of $\alpha/2$ degrees alluded Hamilton, however,
% and  this meant that rotations could only be handled if the vector to be rotated was in the same plain as the rotation.
% This made the algebra complicated, 
% and cause Gibbs to break apart Quaternian into the  $i$, $j$, $k$ axes and a scalar.
% For more on this history see the books of David Hestenes\cite{Dynamicsbook, Hestenes1984} and the Cambridge geometric algebra group\cite{Doran2003}.
% }.
% The differential of the transformation is $\fbar(a) = \Rt a R$ 
% the adjoint is $\fadj(a^\prime) = R a \Rt$ and the determinant is unity.
% Using $R\Rt =1 $ it follows that 
% \begin{align}
%   A^\prime(x) = R A(\xp) \Rt = A(x).
% \end{align}
% Using \eqnref{delTransformation} it follows that 
% \begin{align}
%  \del \wedge A = R \del^\prime \wedge A(x^\prime) \Rt = R F(\xp) \Rt.
% \end{align}
% The vector 
% \begin{align}
% \delta A^\prime &= \frac{d}{d\alpha}R A(x^\prime) \Rt \\
% &= B\cdot A + R \frac{d \xp}{d \alpha}\cdot \del A \Rt \\
% &=  B\cdot A - \lr{B \cdot x }\cdot \del A.
% \end{align}

% Using \eqnref{dL} we obtain the conserved current, the angular momentum 
% \begin{align}
%   \J(B) &= \lr{B\cdot A - \lr{B \cdot x }\cdot \del A}\cdot F+ \half B\cdot x F\cdot F\\
%   &= \lr{B\cdot A} \cdot F + T(x\cdot B)
% \end{align}
% where \eqnref{TEM} has been used.
% The adjoint is 
% \begin{align}
%   \adjoint{\J}(n) &= \del_B\scalar{\J(B) n} \\
%   &= \bivector{A F\cdot n - x\wedge \del T(x) \cdot n} \\
%   &= A \wedge \lr{F\cdot n} +  \adjoint{T}(n) \wedge x
% \end{align}
% The second term  $\adjoint{T}(n) \wedge x$ is the orbital angular momentum of the sound pulse as
% moves through space.
% The first
% \begin{align}
% S(n) = A \wedge \lr{F\cdot n}
% \end{align}
% is the intrinsic spin of the  acoustic field.
% The spin term is often suppressed by putting its influence in the boundary conditions,
% but since we are interested in acoustic plain waves we shall not do this.
% %This has the advantage of making the angular momentum manifestly gauge-invariant,
% %although at the cost of making plain-waves (that reach to infinity) more inconvenient to handle.
% %Again this isn't manifestly gauge invariant.
% %When the fields diminish rapidly at infinity the angular momentum may be  expressed in gauge invarent form,
% %the spin is absorbed into the angular momentum $T(n)$.
% %However, such a procedure is not valid for plain wave solutions (as the field does not vanish at infinity)
% % so we keep \eqnref{AM}.
% %The spin in the lab frame  is
% %\begin{align}
% %  S(v) =A \wedge   \lr{F \wedge \gamma} = -\multi{\phi + \vA \vE}_2 = \phi phi I \vB_v 
%  %  S(n) = A \cdot  \lr{F\wedge  n} =  \lr{A\cdot F} \wedge n %= -\phi I\vB
% %\end{align}
% %The spin in the lab frame  is
% %\begin{align}
% %  S(\g) =A \wedge   \lr{F \cdot \g} = -\phi \vE - I \vA \times \vE  = 
% % %  S(n) = A \cdot  \lr{F\wedge  n} =  \lr{A\cdot F} \wedge n %= -\phi I\vB
% %\end{align}

% \subsection{The acoustical spin as a quantised topological invariant}

% To interpret the spin it is useful to project it onto the direction of motion,
% yielding a scalar known as the helicity,
% \begin{align}
%   \H = \hat \vP \cdot S(\gamma_0),
% \end{align}
% where  $\hat \vP$ is the unit  Poynting vector, 
% \begin{align}
%  \hat\vP =\frac{\vE \times \vB}{\abs{\vE\times\vB}}.
% \end{align}
% and $S(\gamma_0)$ is the timelike component of the Spin 
% as measured in the laboratory frame,
% \begin{align}
%  S(\gamma_0) % &=  A \wedge(F\cdot)  \\
%  &= \half\bivector{A\g \g \lr{F\g - \g F}}\\
%  &= \phi\vE - \vA \times \vE.
% \end{align}
% The helicity is therefore
% \begin{align}
%   \H =  \hat \vP \cdot \lr{\phi \vE - \vA \times\vE} =  \frac{\abs{\vE}^2}{\abs{\vE\times\vB}}\vA \cdot \vB.
% \end{align}
% For a sound wave
% %\begin{align}
% $\abs {\vE} = c\abs{\vB}$
% %\end{align}
% and so the helicity simplifies to
% \begin{align}
%   \label{eqn:Helicity}
% \H = \vA \cdot \vB.
% \end{align}
% %On the left is the electromagnetic helicity. 
% %On the right is the acoustical helicity first defined by Moffat \cite{Moffatt1969}.

% %For the acoustic wave to be transverse in vorticity the particles being perturbed must contribute to  an angular momentum (spin) when the sound wave passes through.
% %We must therefore imagine the particles to be moving on small vortex lines, 
% %rather than `to and throw' in a linear fashion.
% %These loops must be closed, for else momentum will be passed to the fluid  after the sound wave has passed,
% %and the the longitudinal oscillation should coincide with the frequency of the wave. 
% %A motion such as the streamline in \figref{particle} could be imagined.



% %The loop drawn in \figref{particle} may seem overly complicated.
% %The reason that a simple circle, for example, 
% %would not do is due to the non-trivial topology implied by the non-vanishing helicity.
% While $\H$ in \eqnref{Helicity} is defined in terms of the spin,
%  the  term $\vA \cdot \vB$ on the right hand side is the same as the 
% relativistic generalisation\footnote{ \label{footnote:dimensionallity_footnote}
% The relativistic generalisation is accomplished by replacing $\vv$ in Moffatt's definition with $\vA$.
% Notice, furthermore, that Moffatt did not normalise the Helicity,
% that is $\H_{\text{Moffatt}} = p\cdot S$ rather than $\frac{1}{\abs{p}}p\cdot S$,
% where $p$ is the momentum.
% When we make the comparison to $\H = \epsilon_0 A\cdot B$, therefore,
% we implicitly divide $\H_{\text{Moffatt}}$ by a unit momentum,
% so that dimensionally all is correct.
% } to the
% {\em hydrodynamic helicity} per unit volume defined by Moffatt\cite{Moffatt1969}.
% It is readily interpretable.
% $\vB = \vdel \times \vA$ is measures a rotation,
% and the axis of rotation is set by $\vA$.
% The helicity therefore measures the degree to which the fluid streamlines rotate about themselves,
% that is, the degree to which the streamlines are helical\cite{Moffatt1969, Ramada1992}.
% %In a small volume $dV$ the flow is comprised of a superposition of a uniform flow $A_0$, a shear
% %and a rigid body rotation about the origin of $dV$ with an angular velocity of twice the vorticity $2\vB_0$
% %The streamlines of the flow within the sound pulse will be helices about the sheared $A_0$
% The contribution to $\vA \cdot \vB dV \approx \vA_0\cdot \vB_0 dV$
% is positive or negative depending on the orientation of the helix\cite{Moffatt1969}.

% In this way Moffatt identifies the component of the spin parallel to the momentum
% with the {\em orbital angular momentum} of a fluid streamline about its mean trajectory.
% The equivalence between the two helicities (acoustic and hydrodynamic)
% implies that this interpretation of the spin is valid.



% The hydrodynamic helicity was introduced by Moffatt because 
% \begin{align}
%   I = \int \abs{dV} \vA \cdot \vB
% \end{align}
% is an invariant that defines the degree of knottness of the fluid.
% Moffatt had a vortex rings in mind.
% For the two vortex filaments in \figref{}, for example, $\H =2 \alpha \kappa_1 \kappa_2$ 
% where $\kappa_i$ are the circulations around the two circuits $C_i$ of the two vortex filaments,
% \begin{align}
%   \kappa_i = \oint_{C_i} \vu \cdot d\vect{l},
% \end{align}
% where $\vu$ is the velocity around $C_i$,
% and $\alpha$ is an integer that defines the linking of the two filaments.
% If they are unlinked $\alpha = 0$, whereas if they are singularly linked $\alpha = \pm 1$.
% The helicity in a volume, the component of the spin parallel to the momentum, is therefore quantised
% in accordance with the topology of the streamlines.

% \subsection{Interpretation of a sound pulse}

% An acoustic plain wave is  a  perturbation of the fluid particles in the direction of propagation of the pulse.
% Within the perturbation the vorticity vanishes
% and so it is temping to state that $\vB = 0$, innihilating the helicity with it.
% However, this view is not correct.
% To see this we imagine a  wave confined to a narrow tube of radius $r$.
% The plane wave can then be constructed from many  such tubes (see \figref{tubes}).

% At the boundary of an isolated tube slipping occurs between the translating particles that carry the sound
% within the tube
% and the bulk fluid outside.
% This results in a {\em vortex-sheet}\cite{Howe} on the boundary of the tube.
% Also at the boundary there is a discontinuity in the enthalpy $\phi$, 
% carried over from the velocity contribution of $\gamma$ in \eqnref{}.
% The Corriolis acceleration, from  \eqnref{}, therefore has two components.
% A component parallel to the tube $\vE_\parallel$ from the $\dot A$ term and a perpendicular component $\vE_\perp$ from the $\vdel \phi$ term.


% which contains 
% The vorticity from adjacent tubes cancel, 
% but the vorticity at the boundary never does.
% From equation \eqnref{} (Crocco's equation, in the hydrodynamics literature)
% The vorticity on the boundary induces a Coriollis acceleration,
% we see that on the boundary the vorticity 

 
% and it is this sense that a sound pulse can be considered as a transverse Coriollis-voriticity wave.
% The transversivity must can be set in two directions which can be described as being in either a {\em left-}  or {\em right-handed polarisation state},
% and so the helicity must be in one of two states.


% Nevertheless, for the sound pulse not to have a preferred direction, 
% and therefore to conserve angular momentum,
% it is required that it the particle  streamlines form a knot over this infinite interval.
% Perhaps the simplest vortex  knot that may be considered is the trefoil knot illustrated in \figref{}.
% Topologically the trefoil knot is equivalent to two linked rings,
% each  with the same circulation, $\kappa$, as the knot.
% The helicity of the trefoil knot is accordingly $\pm2\kappa^2$.


% The helicity of a photon is $\pm\hbar$.
% If sound is knotted with the simple trefoil knot then the {\em acoustic Plank's constant} would be
% \begin{align}
%   \hbar = 2\kappa^2,
% \end{align}
% (divided by a unit momentum, see footnote \ref{footnote:dimensionallity_footnote}).

% Why $\hbar$ and therefore $\kappa$


% Moffatt's topological interpretation of helicity has been used before to  interpret the quantisation
% of the spin in electrodynamics.
% See, for example the extensive investigations into the {\em magnetic helicity} of Trueba and Ra{\~n}ada\cite{Trueba1996, Trueba2000, Ranada2002}
% (The magnetic helicity is equation \eqnref{Helicity} where the $\vA$ and $\vB$ have there electromagnetic interpretation).
% %This in fact continues the very old notion of the vortex atom
% %that goes back to the very formulation of electrodynamics.
% What is new
% in this thesis is the equivalence of electromagnetic field and the acoustical field.
% The hydrodynamic helicity is the {\em same} as the acoustic helicity;
% the  sound pulse has a quantised spin
% and  this spin may be interpreted, via the  helicity,
% as the orbital angular momentum of  the fluid particles about their mean flow.
% %\cite{Moffatt1969, Moffatt1988, Chechkin1993,  Trueba1996, Trueba2000}.




% %Moffatt demonstrated that the helicity is a measure of the degree of linking of a vortex line.
% %For a knot
% %the total helicity in a volume is an integer multiple of the number of links 
% %that exist when the  vortex line is unknotted.
% %The spin is a topological invariant, and is accordingly quantised
% %The sound field has a  quantised angular momentum, a property  not put in `by hand'.
% %We shall not persue further here the links between electromagnetism, topology and quantisation,
% %but rather refer the interseted reader to the literature 



% %Much of the interest in hydrodynamic helicity stems 
% %comes from its application to magneto-hydrodynamics.
% %Here we note a similarity between an acoustically observed fluid and a perfectly conducting fluid.
% %Both obey  $\dt \vB = \vdel \times \lr{\vv \times \vB}$,
% %which is one of Maxwell's equations in the present scheme.
% %This present discussion is but one example of what ultrasound 
% %can learn much from the magneto-hydrodynamics literature.


% % \begin{figure}[h]
% %  \centering
% %  \includegraphics{trefoil_knot.pdf}
% %  \caption{
% %    A possible streamline (red) with its averaged rotation in green.
% %    The streamline was chosen to be a trefoil knot. The reasons for the non-trivial topology are discussed in the text.
% %    }
% %    \label{fig:particle}
% % \end{figure}



% \section{The force law for a turbulent  source}\label{sec:int:vortex_interaction}
% The force law for an acoustic source when in the presence of an external vorticity or Coriolis field may be found 
% from the manifestly gauge invariant energy momentum tensor\cite{Doran2003}, equation \eqnref{TAEM}.
% Remembering Maxwell's relation, $\del\cdot F = J$ (equation \eqnref{Maxwell})
% we obtain 
% \begin{align}
%  f = \scope T(\scope \del) = -\half \lr{\scope F\scope \del F + F \del F} = -\half \lr{-JF +FJ} = J \cdot F.
% \end{align}
% If the acoustic source has a constant rest mass, $m_0$ and moves at a speed $u$,
% then, by writing  $J = q u$, where $q$ is the acoustic source, we obtain
% \begin{align}
%   f = m_0 \dot u = q u \cdot F, \label{eqn:LFL}
% \end{align}
% Equation \eqnref{LFL}  is the Lorentz force law.

% To write equation \eqnref{LFL} in its more conventional form we project into the laboratory frame,
% \begin{align}
% J\cdot F  = \scalar{\lr{\rho + \vJ} \g\lr{\vE+I\vB}} = -\lr{J\cdot E + q \vE + q \vu\times\vB }\g.
% \end{align}
% The timelike $q\vu\cdot\vE $ is the work done.
% The remaining spatial component is
% \begin{align}
%    \vf = -q \lr{\vE + \vu \times \vB},
% \end{align}
% which is the Lorentz force law in its usual form.

% Equation \eqnref{LFL} describes how a turbulent source moves 
% in the presence of external flows when measured with ultrasound.
% Its linearity is an example of the general rule that the world appears more simple
% when it is measured acoustically.
% This is because, due to the limitations on what ultrasound can measure, reassigns the temporal and spatial locations of entities so that 
% the non-linearity in their motion vanishes.

% % The divergence of the energy momentum tensor gives the force
% % and the spatial component in the lab frame is
% % \begin{align}
% %   \vf = - \lr{\rho_q\vE + \vJ \times \vB}.
% % \end{align}
% % If it is assumed that an acoustic charge $q$ is carried by a `particle' moving at speed $\vu$, then
% % $\vf = - q\lr{\vE + \vu \times \vB}$, which is the Lorentz force law.
% % Turbulent sources of sound, therefore, interact according to the Lorentz-force law when measured acoustically.

% %\subsection{Particles with Spin}

% \section{Further work: Interactions between particles with spin}\label{sec:int:electron}

% The (exact) analogy developed in this thesis between acoustics and electromagnetism
% is incomplete in that we are missing an analogue to the electron:
% a particle with spin that  interacts with the spin of the sound pulse.
% We have found that turbulance is the source of the sound
% field - a role played by the electron for electromagnetic waves - 
% but we have not found an analogy of the electron itself.
% %Here we suggest path  for completing the  analogy.
% %We hope that it will  provide be a useful start to future research.

% In \secref{} we found that the spin parallel to the momentum 
% can be interpreted as resulting from  helical trajectories of the fluid particles through spacetime,
% with the spin resulting from the  orbital angular momentum of the fluid precession  about a mean trajectory.
% For this helicity not to integrate to zero,
% Moffatt demonstrated that the trajectory of the fluid particles must be knotted in some manner.
% The acoustic electron is presumably, therefore, comprised of a confined region of fluid undergoing some form of knotted helical motion.


% Interpreting the electron's trajectory as a helix about a mean trajectory is not a new idea. 
% It is at the core of the \zitter\ interpretation of quantum mechanics which dates back to Schr\"odinger.
% It has more recently overhauled by David Hestenes, and put into a form consistent with the Dirac equation\cite{}.
% In the \zitter\ interpretation the spin of the electron is interpreted as the orbital angular momentum of the electron 
% about its centre path.

% %
% %and provides the most promising starting point for our analogy.
% Using the \zitter\ as our starting point,
% we identify the electron with a region of the fluid undergoing helical motion.
% We do not assume that the actual fluid particles within the region are fixed,
% a given particle may flow through the region.
% We then use consistency with the Dirac equation to impose constraints upon this trajectory.
% As might be expected, the constraints are stringent and we are unable imagine how such a streamline can arise
% in acoustics.
% Nevertheless,
% it does give a beginning on which to build.

% \subsection{Correspondence with the \zitter\ model}

% The requirements for \zitter\ to be consistent with the Dirac equation have been catalogued.
% Here we discuss the list given in David Hestenes' article on self-interaction\cite{},
% \nlist{
%   \item the acoustic electron is a massless point particle. \label{index:massless}
%   \item the orbital-angular momentum, or spin, about the helix is fixed at $s=\hbar/2$, \label{index:spin}
%   \item the frequency of the precession is the de Broglie frequency $\omega_0 = mc^2/\hbar$,\label{index:freq}
%   \item the acoustic electron has a charge $e$
%   \item the total energy of the acoustic electron is $mc^2 = E_0 + U_0$ where $E_0$ is the kinetic energy of the helical trajectory,
%     and $U_0$ is the self-interaction potential.
% }
% From property \ref{index:massless} it follows that the helical trajectory travels at the speed of sound.
% This is not to say that the acoustic-electron travels at the speed of sound, 
% however, 
% for the projection onto the mean path will be timelike.
% From property \ref{spin} we find that the spin of the acoustic wave is constrained to be half that of the sound pulse.

% The radius of the helix is (from \ref{index:freq}) 
% \begin{align}
%   r = \frac{c}{\omega} = \frac{\hbar}{mc}
% \end{align}
% From property \ref{index:spin} it follows that
% %\begin{align}
%   $s = r\frac{E_0}{c} = \frac{\hbar}{2}$
% %\end{align}
% and so 
% \begin{align}
%   E_0 = s \omega_0 = mc^2/2
% \end{align}
% from which it follows that $E_0 = U_0 =  mc^2/2$.

% The magnetic moment - or vorticity moment - 
% is 
% \begin{align}
% \mu = \frac{ec}{2\pi r}\frac{\pi r^2}{c} = \frac{er}{2} = \frac{e}{mc}s = \frac{e\hbar}{2mc}
% \end{align}

% The Dirac current is then interpreted as the most probable mean trajectory 
% and the spin current is interpreted as the most likely direction of the spin at each spacetime point.

% \subsection{Discussion}


% To pursue the analogy between the electron and acoustics we need an algebraic system that unifies as much as possible
% classical and quantum physics.
% David Hestenes geometric algebra does this very elegantly.
% As has been forcefully demonstrated by Hestenes\cite{HestenesMechanicsBook} 
% and Doran and Lasenby\cite{Doran2003},
% much of the `quantum mathematics' has nothing intrinsically tied to quantum mechanics, 
% but rather describes mechanics in a 4-dimensional spacetime.

% Using this algebraic system
% we find that many of the features that we wish to 
% incorperate into the vortex-with-helicy-sound interaction are already present 
% in David Hestenes Zitterberegung model of the electron,
% which we use as a template.
% The  Bjerknes model of pulsations suggests an interpretation to the 
% electron's interaction parameter $\beta$, 
% which is the only parameter without adequate interpretation in the Dirac-Hestenes model.

% While the analogy between a pulsating bubble and an electron is far from satisfactory,
% it is true to say that a pulsating bubble has most of the ingredients required.
% It therefore makes an interesting first attempt, and a good point of departure for further research.





% Vortex rings with swirl,
% by virtue of their angular momentum about their symmetry axis, have been used as  classical models of spin \cite{Pekeris1953}.
% It is, however, the link between topology, helicity and spin that makes the analogy more than a useful conceptual aid.
% Moffatt \cite{Moffatt1969}, for example, when considering Hicks' spherical vortex notes  that 
% ``every torus knot is represented once and only once amongst all the vortex lines of each member of the family of flows described by the stream function''.
% Accordingly, the intrinsic angular momentum of such toroids 
% can be at least partially identified with its spin.
% %The spin will interact with an external vorticity field with a magnetic moment type reaction.
% %By appeal to the electromagnetic analogy we call the strength of the interaction the {\em magnetic moment}.
% %Since there is no danger of confusion, it seems better to use the terms of the analogy rather than introducing names such as {\em vorticity-moment}.
% %The magnetic moment is $\mu = \gamma_g \vs$ where $\vs$ is the spin vector and $\gamma_g$ is the giromagnetic ratio.
% %The energy of the interaction is $H = \mu \cdot \vB$. % and accordingly the force is $-\vdel H =-\vdel  \mu\cdot \vB$.
% %The spin-vorticity interactions need to be added
% %to the Lorentz force law of \eqnref{LorentzGA}.


% The kinematics of a vortex ring can be modelled by placing a local orthonormal coordinate frame $\{e_0, e_1, e_2, e_3\}$ at the centre of the ring.
% The vector $e_0$ is timelike and is tangent to spacetime path of the centre of the vortex, $x(\tau)$.
% Then, with $\tau$ being the proper time, $e_0 = \frac{dx}{d\tau} = u$, where $u$ is the velocity, and $e_0^2 = 1$.
% The three orthogonal axes are spacelike, so that $e_i^2 = -1,$ for $i = 1,2,3$.
% The vector $e_3 = s$ is oriented so that it is  orthogonal to the vortex ring, and is called the spin vector.
% The vectors, $e_2$ and $e_1$ define the {\em spin-plain}, $S = e_2e_1$ on which the vortex ring is confined.
% The vector $e_2$ is set to point towards the average motion of a particular swirling particle (following the green line in  \figref{particle})
% enabling the magnitude of the spin to be captured.

% %If the vortex is moving inertially, its momentum $p$ will be constant.
% %%$m$ is a measure of the vortex's intrinsic mass, perhaps the mass difference displaced by any pressure gradients within the vortex.
% %By defining and effective mass, $m =  p \cdot u$ and integrating, it follows that  $p\cdot(z - z_0) = m\tau$.
% %If the angular momentum is conserved then $lr{z - z_0} \wedge p = S(\tau) - S_0$ and so the equation of motion follows
% %\begin{align}
% %z = (S(\tau) - S_0) \cdot p^{-1} + m p^{-1} \tau + z_0 = x(\tau) + r(\tau),
% %\end{align}
% %where $x(\tau) = S_0 \cdot p^{-1} + m p^{-1} \tau + z_0 $ is the spacetime path of the central line, 
% %and $r(\tau) = S(\tau)  \cdot p^{-1}$ is the radius of the vortex ring.
% %The particle that is followed by $e_2$ is then seen to follow a helix of radius $r$ in spacetime.
% Although it is not immediately obvious,
% the model presented so far is essentially identical to the `timelike' case of the David Hestenes zitterbewegung interpretation of 
% electron physics 
% \cite{Hestenes1973, Hestenes1990,  HestenesResearchProgram}. 
% We apply the results of these papers freely, 
% although argue the results in terms of the meaning to acoustically measured turbulence. 


% The motion of the reference frame $\{e_\mu\}$ models the motion of the vortex
% and can be obtained from a constant reference frame, $\{\gamma_\mu\}$ by means a Lorentz rotation.
% Rotations are double sided transformations in geometric algebra, 
% and is written,
% $e_\mu = R \gamma_\mu \Rt$,
% where the {\em rotor} $R$ is an even multivector with reverse $\Rt$, such that $ R\Rt = 1$.
% Since geometric positioning of $\{e_\mu\}$ is all that is modelled,
% the kinematics of the frame can be completely determined by finding the
% $\tau$-dependent bivector $\Omega =  2\dot{R}\Rt $, which represents the angular
% velocity of the frame,
% %kinematics is expressed entirely by the rotor $R$,
% %such that
% %\eqal{
% %\dot{R} = \half\Omega R.
% %}{RotorEqn}
% %Alternatively, the influence on each axi
% \begin{align}
%   \dot e_\mu = \Omega \cdot e_\mu
% \end{align}
% The Lorentz force in \eqnref{LorentzGA} is of this type, with $\OmegaEM = \frac{q}{m_0}F$.
% %where $\Omega =  2\dot{R}\Rt $ is an 
% %The dynamics of the frame can be completely determined by finding the
% %$\tau$-dependent bivector $\Omega =  2\dot{R}\Rt $, which represents the angular
% %velocity of the frame.

% The energy of the fluid in the vortex is modelled by the rotation of $e_2$ in the spin plain $S$.
% Therefore, energy-momentum of the vortex is described by the rotation in this plain, $\Omega \cdot S$.
% %The quantisation of the spin in \secref{lagrangian} came from the knottedness of the flow around the spin-plain, 
% %which implies that $\oint \lr{ dx^\mu \Omega_\mu} \cdot S = nk$,
% %where $k$ is the vortex strength and $\Omega_\mu$ is the acceleration in the direction of $\gamma_\mu$.
% To include interactions with sound wave, 
% %If the vortex interacts with sound then for the angular momentum to be both conserved and obey the quantisation,
% it should be the  {\em canonical energy-momentum}, $P = p - eA$, that is equated to $\Omega \cdot S$.
% %It is appropriate that the sound contributes its energy to the spin, 
% %for this is the only way in which angular momentum is conserved. 
% Therefore
% \begin{align}
%   \Omega\cdot S = P \cdot u = m - eA\cdot v.
% \end{align}
% If a sound wave deposits $\OmegaEM \cdot S$ of energy to the vortex ring, then its mass must also increase.
% %The  $m = p \cdot u$ cannot therefore be equal to the free vortex mass $m_0$,
% %but must be
% Therefore
% \begin{align}
%   m  = p \cdot u =  \Omega \cdot S + eA\cdot v= m_0 + \OmegaEM \cdot S + eA\cdot v.
% \end{align}
% The other components of the angular velocity in the direction orthogonal to the spin plain is labelled $ v\cdot  ( I q) =- v\cdot (\Omega \wedge S)$
% while the rate at which the spin plain changes direction as it moves round its axis is $\dot S = v \cdot \del S = \Omega \times S$.
% Note that the symbol $\times$ here represents the commutator product rather than the Gibbs vector product.
% Bringing all the components together
% \begin{align}
% \label{eqn:OmegaS}
% \Omega S = \Omega \cdot S + \Omega \times S + \Omega \wedge S  = v \cdot P + \dot S + Iq
% \end{align}

% An explicit expression for $\Omega$ can be obtained by setting
% \begin{align}
%   \label{eqn:OmegaS2}
%   \Omega S &= m_0 + \OmegaEM S && \implies & \Omega &= m_0 S^{-1} + \OmegaEM = m_0 S^{-1} + \frac{e}{m_0} F.
% \end{align}
% Clearly then \eqnref{LorentzGA} is satisfied.
% Also it follows that  $\dot S = \frac{e}{m_0} F \times S$ which gives an expression for the precession of the spin plain due to the magnetic moment.

% Combining \eqnref{OmegaS} and \eqnref{OmegaS2} gives
% \begin{align}
%    v \cdot p- eA\cdot v  + v \cdot \del  S - v\cdot  (I q)=m_0 + \frac{e}{m_0} F S = m - eA\cdot v 
% \end{align}
% This is the classical approximation  of the Dirac equation in the absence of statistical terms \cite{Hestenes1990}.
% The full  equation in the absence of statistical terms is 
% \begin{align}
%  p -Iq +  = mv - \del S.
% \end{align}
% The similarities and differences  between these equations is found in the literature \cite{Hestenes1990}.


% A simple example of bound hydrodynamic helicity is Hick's spherical vortex\cite{Moffet1980}.
% This model is particularly interesting to us for it also 
% models the flow of a gas within a large bubble, where the surface tension can be neglected.
% Pulsations can futher be incorperated in an elementary way by 
% adapting the technique of Carl and Vilhelm Bjerknes to separate the interaction term caused by the pulsation 
% from the underlying force law.
% A second virtue of Hick's votex is that the angular momentum is conserved when the speed of sound is
% equal to the speed of light, which is exactly our measurement condition.
% Indeed, Pekerisis found that this conservation law applies {\em only} when the sound speed equals the speed of light.
% Pekerisis had no physical reason to impose this condition on the world, 
% but nevertheless considered the possibility of using Hick's vortex to model a neutron.
% %With the conditions of acoustic measurement considered in \chapref{measurement},
% %such a calculation seems entirely natural.
% \section{Modelling an oscillating Bubble}\label{sec:bubble}
% %Micron-sized bubbles are resonant at medical ultrasound frequencies and are used as contrast agents
% %for they  greatly enhance the signal  from blood which is generally echo-poor.
% %To consider their acoustic output it is often sufficient to ignore turbulence and simply calculate the oscillations of a monopole source.
% %However, their {\em motion} will generally not be free from turbulence effect - indeed, the microbubbles are often used to {\em measure} such turbulence.
% %It would be nice to bring their interactions with sound into the general framework presented so far.
% %To do so, the bubble is assumed to be an enclosed vortex. 
% %Toroidal bubbles bound in a region of vorticity have been observed many times - and even generated by Dolphin's in play \cite{}.
% %A spherical vortex with swirl is stable even when stationary in the fluid - and it is this flow that we imagine within a bubble.
% %In the model presented, the mass of the vortex increases when it absorbs a sound wave.
% %This models the volume fluctuations only at their most crude,
% %although for the purpose of dynamics this may be sufficient.


% To adequately model the dynamics of pulsating body such as a bubble, the  {\em interaction parameter}, $\beta$  needs to be introduced into the model.
% This can be done by substituting  $F \rightarrow F e^{I\beta}$ in the model above.
% Since the bubbles are very small, we additionally introduce a probability density $\rho$ for their location 
% - where the symbol $\rho$ should not be confused with the mass or charge density.
% Inserting these into the model above results in
% \begin{align}
%    \rho v \cdot p e^{-I\beta}  + v \cdot \del  \lr{\rho S e^{I\beta}} - \rho v\cdot  (I q)= \rho m \cos\lr{\beta}
% \end{align}
% This is the classical limit to the full Dirac equation including `statistical factors' \cite{Hestenes1973}
% The interaction intensity $\beta$ is uninterpreted in Dirac theory, although it is known to be important for determining whether an election is in a particle or anti-particle state.
% Here $\beta$ plays a similar role (it determines whether the force is attractive or repulsive).
% Whether the identification stands up to scrutiny remains to be seen.
% \section{The acoustic analogy to the electron}\label{sec:int:electron}


% %The derivation is most convenient when using David Hestenes geometric algebra\cite{} and will be used here.
% %This is because only in this algebra can Maxwell's relations be expressed in a single equation.
% %Furthermore, the vector notation of Geometric algebra should be familiar even to non-afficienardos,
% %and so its use should not be a hinderence to meaning.


% Acoustic sources will, in general, interact with one another. %, with this interaction  mediated by sound.
% %The study of vortex-vortex interactions has a  long history  \cite{Whittaker1951},
% The study of the interactions between pulsating bodies
% is of  interest in medical ultrasound because pulsating micron sized bubbles are used as contrast agents.
% The induced motion  between neighbouring bubbles has been photographed
% and the force between them is often  modelled according to Bjerknes' law  \cite{Crum1971}.
% This states that the average force on a bubble, $\scalar{f}$, is
% equal to volume displaced by the bubble, $ V$, multiplied the acceleration 
% induced by a pressure gradient, $\vdel p$ \cite{Bjerknes1905,Crum1971, Leighton1990},
% \begin{align}
%   \label{eqn:Bjerknes}
%   \scalar{f} = \scalar{V(t)\vdel p(\vx, t)}.
% \end{align}
% The  average force  over an oscillation cycle will not be zero if the bubble do not pulsate in phase with the sound wave.
% % f the oscillations arThe phase is denoted with a 
% % % and the pressure oscillates sinusoidally with a frequency $\omega$, so that  $P(\vx, t) = P(\vx)\sin\lr{\omega t}$.
% % To see that the average can give non-zero results,
% If we assume that the spatial and temporal components of the pressure may be separated, 
%  so that  $P(\vx, t) = P(\vx)\sin\lr{\omega t}$,
%  and assume that the bubble's pulsations are small, 
%  so that $V(t) = V_0\sin\lr{\omega t + \beta}$,
% then 
% $\beta$ is the phase of the bubble's oscillations in comparison to the driving sound wave.
% %Then 
% %\begin{align}
% %  \scalar{F} = V_0\vdel P(\vx) \scalar{\sin(\omega t+\beta)\sin(\omega t)} =  \frac{1}{2}V_0\vdel P(\vx) \cos(\beta)
% %\end{align}
% The resultant force over a cycle of oscillation is 
%  $\scalar{F} = \frac{1}{2}V_0\vdel P(\vx) \cos(\beta)$.
% %The strength of the force is determined by $\cos(\beta)$,
% %which can be either positive or negative.

% The  cost of this simple model is that the equations describing the surrounding fluid are linearised.
% Vorticity is not considered and we are entirely removed from the general problem of 
% turbulence-sound interaction.
% This is a shame, for while it is true that the acoustic output of a pulsating bubble far exceeds that of turbulent sources,
% turbulence is still excepted to have an influence on the bubble's motion.
% %Bubbles will not only be carried by turbulent flow but will also interact with the sound generated by it.

% Bjerknes' law, in itself, assumes little about the bubble. 
% It requires only a pulsating volume.
% The details of the bubble, such as its surface tension, are hidden in the parameter $\beta$ that specifies how close to resonance the bubble is \cite{Leighton1990}.
% The motion and interaction of a bubble, therefore, 
% can be firmly set within the context of turbulence-sound interactions, 
% so long as the bubbles mass varies when the bubble interacts with an acoustic field.
% This chimes with the attempts to use Hicks'  spherical vortex as a model of a bubble \cite{Levine1959}.
% Indeed, so long as the bubble is small compared 
% to the wavelength of the sound with which it interacts, then any vortex should do.
% %Bjerknes' law not require us to be too specific.


% In this report, as a means of investigating the interactions between bubbles and sound in its full context,
% models to describe sound-turbulence interactions in general are suggested.
% To simplify the approach we consider only the special case of when the interactions are measured acoustically,
% such as is the case when the turbulence is measured with ultrasound.
% %The formulation therefore remains general, 
% It turns out that  acoustics  is greatly simplified when imaged with sound.
% In fact,
% as is shown in \secref{measurement},
%  Lighthill's formulation of acoustics reduces to an {\em exact} analogue of electromagnetism.
% The reasons for this are briefly discussed in \secref{measurement} 
% and in more detail in a companion paper elsewhere in these proceedings.

% To find the physical consequences of the equivalence,
% the analogy is used to define an acoustic Lagrangian.
% %The exactness of the analogy has a number of consequences.
% %To find them, it is noted that since the field equations are the same, 
% %the acoustic field and the electromagnetic field 
% %can be characterised by a common Lagrangian.
% This is done in \secref{lagrangian} and the implications to angular momentum and helicity are discussed.

% Starting from the Lagrangian, the acoustic energy momentum tensor follows,
% from which the required interaction laws between sound and turbulence can be found.
% Unsurprisingly, in light of the analogy found, 
% this is the Lorentz force law  (\secref{spinless}).

% Swirl, which is  the motion along a vortex ring, for example, is harder to incorporate.
% This is because, as argued in \secref{spin}, a vortex with swirl has an intrinsic angular momentum
% which will be involved in the interaction.
% To simplify, the size of the vortex is reduced to a point.
% Its swirl and velocity can then be modelled with a co-moving frame of reference,
% %One axis of the coordinate frame traces the path of the vortex,
% %another traces the rotion.
% %The helical path through spacetime of any given particle within the vortex can then be traced.
% %This is discussed in \secref{spin}.  
% The interaction term $\beta$ is included in \secref{bubble},  enabling bubble motion to be modelled.
% The model chosen has a great deal of common with classical models of quantum  spin.
% Our thinking has been particularly influenced by David Hestenes' timelike  Zitterbegung model of the election \cite{Hestenes1973, Hestenes1990, HestenesResearchProgram}.  
% It is interesting how naturally the zitterbewegung model can be applied to turbulence.

% Finally, we note that the geometric nature of the models discussed is most straightforward
% and  concise when expressed in the {\em Spacetime Algebra} of David Hestenes \cite{Hestenes2003}.
% %This notation is sufficiently close to usual 3-dimensional vector algebra to be familiar,
% %but unfortunately space here does not permit an introduction to the algebra.
% Important results will be projected into the laboratory frame,
% so that they can be understood entirely with familiar 3-dimensional vector algebra.
% We use the convention that three dimensional vectors are displayed in bold.
% This helps to distinguish them from their spacetime analogues.

% \section{Discussion}\label{sec:discussion}

% We have constructed a general model for the dynamic interactions between sound, turbulence and bubbles.
% If the model is confirmed experimentally,
% then it represent a  relatively simple yet powerful framework in which to investigate the motion and dynamics of acoustic sources.

% In order to be able to construct the models the following observations were necessary:
% \nlist{
% \item  That the measurement process used in acoustical measurement is identical to that used in optical relativistic theories,
%     but with the speed of sound taking the role of the speed of light.
% \item  That acoustics formulated on an incompressible, relativistic ideal fluid (which embodies measurement considerations)
%     is identical to classical electromagnetism.
% }
% The authors are unaware of these observations having been made before.

% Having made the link between electromagnetism and acoustics, 
% for which a relativistic theory is  a pre-requisite,
% the application to acoustic interactions followed more-or-less mechanically.
% This indicates, on the one hand, the potential power of the analogy.
% It represents, on the other hand, a set of stringent experimental test with which to test the link between acoustics, relativity and electromagnetism.



% \section{Discussion}



% %%%%%%%%DISCUSSION
% The correspondance is exact, and successfully unifies the much observed the similarites
% between the magnetic field and vorticity (which was used by Maxwell)
% and the electric field and the corriolis acceleration (\cite{Marmanis2000,Sridar1998})
% Marmanis\cite{Marmanis} and Sridar\cite{1998} analogies were only partial 
% due to their application of the Gallilean invariance rather than
% the Lorentz invariance appropiate for electromagnetism and ultrasound.
% %%%%%%%%%%%%




% We have shown that Maxwell's equations may be derived from a relativistic (Lorentz invarient) ideal fluid,
% where the speed of sound takes the role of the speed of light.
% The Coriolis acceleration - also known as the Lamb vector - takes the role of the electric field 
% and that the vorticity takes the role of the magnetic field.
% The relativistic vorticity bivector takes the role of the field tensor.
% This identification of terms was made before by Mermanis\cite{Mermanis}.
% However, their attempt started from the Gallilean invarient approximation 
% and so Maxwell's equations could not be written down while still maintaining sound as the propergating medium.
% This work completes their analogy by making it exact.

% At first, however, the analogy derived here is dissapointing.
% A relativistic fluid, where the speed of sound equals the speed of light, 
% seems so esoteric that applications for the analogy seem in short surply.
% There is a major and terrestial application for the analogy, however,
% and that is to ultrasound physics.

% In ultrasound distances are determined from 
% \nlist{
% \item the time interval between a sound pulse leaving the transducer
% and the echo returning,   
% \item the average speed of sound of the medium.
% }
% The average speed of sound of the medium is required before anything can be said about the locations of entities
% in the world.




\section{Discussion}




The derivation followed from noting that for ultrasound to measure distances (using pulse-echo) 
the sound speed in the medium must be known {\em a priori}, 
with no possibility of measuring variations in this sound speed.
This has two consequences:
\begin{enumerate}
  \item The sound speed must be measured to be a constant.
  \item The sound must be measured to be propagating linearly.
\end{enumerate}
If invariance to inertial translations is also assumed,
then the constancy of the sound speed  implies that 
ultrasound is subject to the considerations of special relativity,
with the sound speed taking the role of the speed of light.
This invariance has been assumed here.

To describe the propagation of the sound we considered an ideal fluid medium.
When measured acoustically the relativistic (Lorentz invariant) description of the ideal fluid should be used,
with the additional constraint that the speed of sound equal the speed of light.
We have shown that the sound does indeed propagate linearly in this case.
Indeed, we have shown it obeys Maxwell's relations.


It is instructive to see this argument borne out in the equations that describe an acoustically measured fluid.
The equations of motion must  be Lorentz invariant and this condition  is automatically fulfilled  when  the equations 
are obtained from the divergence of the energy-momentum tensor of the ideal fluid.
In the derivation we also need to use the hitherto unused  condition
that the sound speed takes the role  of the speed of light.
This condition is imposed by equating the two speeds.
This further requires that the energy density for the fluid, as measured acoustically, be a function of the pressure only (barotripic),
for the sound speed cannot equal the speed of light otherwise\cite{Taub1978}.




\bibliographystyle{plainnat} % apa	
\bibliography{../../tshorrock_thesis/bib/tshorrock}

\end{document}





%%% Local Variables: 
%%% mode: latex
%%% TeX-master: t
%%% End: 
